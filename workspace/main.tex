\documentclass[12pt,a4paper]{report}

\usepackage[backend=biber, style=ieee]{biblatex}
\addbibresource{literature.bib}

\usepackage[utf8]{inputenc}
\usepackage[T1]{fontenc}
\usepackage[ngerman]{babel}
\usepackage{lmodern}
\usepackage{enumitem}
\usepackage{graphicx}
\usepackage{tabularx}
\usepackage{float}
\usepackage[onehalfspacing]{setspace}
\usepackage{microtype}
\usepackage{parskip}

\usepackage{xcolor}
\newcommand{\todo}[1]{\colorbox{red}{\textbf{TODO: #1}}\\}
\newcommand{\question}[1]{\colorbox{yellow}{\textbf{QUESTION: #1}}\\}
\newcommand{\xeno}[1]{\colorbox{pink}{\textbf{TODO XENO: #1}}\\}
\newcommand{\gideon}[1]{\colorbox{green}{\textbf{TODO GIDEON: #1}}\\}

\begin{document}

\tableofcontents
\newpage

\chapter{Konzeptlösung für die Erweiterungen an Yappi}

Die im vorhergehenden Kapitel beschriebenen Problemstellungen und die Analyse bestehender Lösungen verdeutlichen, dass aktuelle
Ansätze zur Erfassung der Entwicklerzufriedenheit häufig unvollständig sind. Sie erfassen den Einfluss einzelner Arbeitsereignisse
nicht systematisch und bieten nur eingeschränkte Möglichkeiten, die erhobenen Daten im individuellen Arbeitskontext zu
interpretieren. Zudem werden potenziell relevante Einflussfaktoren, wie das physische Wohlbefinden, nur selten berücksichtigt.

Mit den geplanten Erweiterungen soll Yappi genau diese Lücken schliessen und eine integrierte, praxisorientierte Lösung
bereitstellen, die Entwicklerinnen, Entwickler und Teams gleichermassen unterstützt. Ziel ist es, die Entwicklerzufriedenheit 
umfassend zu erfassen und die gewonnenen Erkenntnisse in konkrete Handlungsempfehlungen zu überführen.

Die Erweiterung basiert auf folgenden Ansätzen:

\begin{itemize}

    \item Literaturrecherche zum Stand der Forschung im Bereich der Entwicklerzufriedenheit und deren Einfluss auf die
      Produktivität
    \item Analyse bestehender Lösungen zur Erfassung und Auswertung von Entwicklerzufriedenheit, Arbeitskontext und 
      Gesundheitsdaten im Umfeld der Softwareentwicklung
    \item Befragungen von Entwicklerinnen und Entwicklern in den Unternehmen der Autorinnen und Autoren
    \item Brainstorming mit den Projektbetreuenden
\end{itemize}

In den folgenden Unterkapiteln wird die schrittweise Entwicklung der Konzeptlösung für die geplanten Erweiterungen von Yappi
detailliert beschrieben.

\section{Problemstellungen}

Die Erweiterung von Yappi verfolgt das Ziel, bestehende Schwächen in der Erfassung und Interpretation der Entwicklerzufriedenheit
zu beheben. Dazu wurden die relevanten Herausforderungen nach dem Standard des Rational Unified Process (RUP) strukturiert
beschrieben. Jede Problemstellung wird dabei in einem einheitlichen Schema dargestellt, um die Auswirkungen klar herauszuarbeiten
und eine fundierte Basis für die Definition der Projektziele zu schaffen.

Die folgenden Abschnitte fassen die drei zentralen Problemstellungen zusammen, die mit der Erweiterung adressiert werden sollen.

\subsection{Passive Erfassung der Entwicklerzufriedenheit}

In vielen aktuellen Umsetzungen erfolgt die Erfassung der Zufriedenheit rein passiv, d.\,h., Entwicklerinnen und Entwickler
müssen von sich aus aktiv eine Rückmeldung abgeben. Dieser Prozess ist stark von der Eigeninitiative abhängig und wird im
Arbeitsalltag häufig vergessen oder aufgeschoben. Dadurch entsteht eine unvollständige und unregelmässige Datengrundlage, die den
tatsächlichen Verlauf der Zufriedenheit nur eingeschränkt widerspiegelt. Die vollständige Problemstellung ist in 
Tabelle~\ref{tab:rup-passive} dargestellt.

\begin{table}[H]
  \caption{Problemstellung nach RUP: Passive erfassung der Entwicklerzufriedenheit}
  \label{tab:rup-passive}
  \centering
  \begin{tabularx}{\textwidth}{>{\raggedright\arraybackslash}p{0.25\textwidth}|X}
    \hline
    Das Problem & der passiven Erfassung der Entwicklerzufriedenheit,
    bei der ein Entwickler von sich aus eine Rückmeldung abgeben muss \\ \hline

    betrifft & Entwicklerinnen und Entwickler. \\ \hline

    Die Auswirkung dieses Problems & ist eine geringe Erfassungsrate der Zufriedenheitsdaten,
    da Feedback oft vergessen oder aufgeschoben wird. \\ \hline

    Eine erfolgreiche Lösung & fordert Entwicklerinnen und Entwickler zu relevanten
    und regelmässigen Zeitpunkten automatisiert dazu auf, Zufriedenheitsdaten zu erfassen.
    Dabei darf kein grosser Mehraufwand entstehen, um zur Abgabe zu motivieren. \\ \hline
  \end{tabularx}
\end{table}

\subsection{Einfluss einzelner Meetings}

Meetings nehmen einen wesentlichen Teil der Arbeitszeit von Entwicklerinnen und Entwicklern ein. Werden sie als unproduktiv oder
belastend wahrgenommen, kann dies die Arbeitszufriedenheit deutlich beeinträchtigen. Derzeit wird jedoch der Einfluss einzelner
Besprechungen auf die Zufriedenheit nicht systematisch erfasst, was eine gezielte Verbesserung der Meetingkultur erschwert. Die
vollständige Problemstellung ist in Tabelle~\ref{tab:rup-meetings} dargestellt.

\begin{table}[H]
  \caption{Problemstellung nach RUP: Einfluss auf die Zufriedenheit einzelner Meetings wird nicht erfasst}
  \label{tab:rup-meetings}
  \centering
  \begin{tabularx}{\textwidth}{>{\raggedright\arraybackslash}p{0.25\textwidth}|X}
    \hline
    Das Problem & dass der Einfluss einzelner Meetings auf die Zufriedenheit nicht erfasst wird \\ \hline

    betrifft & Scrum Master, Product Owner, Entwicklerinnen und Entwickler. \\ \hline

    Die Auswirkung dieses Problems & ist, dass unproduktive oder belastende Besprechungen schwer identifiziert werden können und
    deren Auswirkungen auf die tägliche Arbeit unbekannt bleiben. \\ \hline

    Eine erfolgreiche Lösung & erfasst nach relevanten Besprechungen zeitnah die wahrgenommene Produktivität und Belastung, um
    den Einfluss einzelner Meetings auf die Arbeitszufriedenheit sichtbar zu machen. \\ \hline
  \end{tabularx}
\end{table}

\subsection{Schwierigkeiten beim Ableiten von Schlussfolgerungen}

Die Erhebung reiner Zufriedenheitswerte ohne weiterführende Informationen erschwert es, deren Ursachen oder Auslöser zu verstehen.  
Ohne zusätzlichen Kontext ist es schwierig, Muster zu erkennen oder Zusammenhänge zwischen bestimmten Arbeitsbedingungen und der
Zufriedenheit herzustellen. Dies führt dazu, dass Massnahmen zur Verbesserung oft auf Annahmen basieren und nicht ausreichend
datenbasiert sind. Die vollständige Problemstellung ist in Tabelle~\ref{tab:rup-conclusions} dargestellt.

\begin{table}[H]
  \caption{Problemstellung nach RUP: Schwierigkeiten beim ableiten von Schlussfolgerungen aus Zufriedenheitsdaten}
  \label{tab:rup-conclusions}
  \centering
  \begin{tabularx}{\textwidth}{>{\raggedright\arraybackslash}p{0.25\textwidth}|X}
    \hline
    Das Problem & dass es schwierig ist, aus den erfassten Zufriedenheitsdaten klare Schlussfolgerungen abzuleiten \\ \hline

    betrifft & Scrum Master, Product Owner, Projektverantwortliche, Entwicklerinnen und Entwickler. \\ \hline

    Die Auswirkung dieses Problems & ist, dass unklar bleibt, welche Faktoren oder Ereignisse zu den gemessenen Ergebnissen 
    geführt haben, wodurch gezielte Massnahmen zur Verbesserung erschwert werden. \\ \hline

    Eine erfolgreiche Lösung & ergänzt die Zufriedenheitsmessungen um automatisch erfasste Kontextinformationen aus dem 
    Arbeitsumfeld sowie ausgewählte Gesundheitsdaten, um die Ergebnisse besser einordnen und deren Ursachen gezielter 
    identifizieren zu können. Zusätzlich leitet sie konkrete Handlungsempfehlungen aus diesen Daten ab. \\ \hline
  \end{tabularx}
\end{table}

\section{Idee}

Die Erweiterung von Yappi verfolgt das Ziel, die Erfassung und Interpretation der Entwicklerzufriedenheit zu verbessern und deren
Aussagekraft zu erhöhen. Grundlage ist die Annahme, dass eine ganzheitliche Erhebung, ergänzt durch Informationen aus dem
Arbeitsumfeld, ein genaueres Verständnis der Einflussfaktoren ermöglicht. In bestehenden Lösungen werden Zufriedenheitswerte oft
isoliert erfasst, was die Ableitung konkreter Verbesserungsmassnahmen erschwert.

\subsection{Automatisierte Aufforderung zur Erfassung von Zufriedenheitsdaten}

Ein zentraler Bestandteil der Erweiterung ist, dass Entwicklerinnen und Entwickler nicht mehr ausschliesslich aus eigener
Initiative Zufriedenheitsdaten erfassen müssen. Stattdessen werden sie zu geeigneten Zeitpunkten automatisch dazu aufgefordert,
kurze Rückmeldungen zu geben. Es wird angenommen, dass diese Form der aktiven Aufforderung die Häufigkeit der Erfassungen erhöht
und somit eine vollständigere sowie aussagekräftigere Datengrundlage schafft.

Ein relevanter Zeitpunkt zur Aufforderung zur Erfassung von Zufriedenheitsdaten sollte so gewählt werden, dass der Entwickler
nicht aus seinem Arbeitsfluss herausgerissen wird. Geeignete Zeitpunkte dafür sind:

\begin{itemize}
  \item \textbf{Nach einem Commit:} Entwicklerinnen und Entwickler übermitteln im Arbeitsalltag regelmässig ihre vorgenommenen
    Codeänderungen an das Versionsverwaltungssystem (Commit). Direkt nach dem Abschluss einer Entwicklungsaufgabe oder eines
    Arbeitspakets wird der bestehende Arbeitsfluss ohnehin kurz unterbrochen, um den Commit auszuführen. Dieser Moment eignet sich
    daher, um eine kurze Rückmeldung zu erfassen, ohne den Entwickler aus einer konzentrierten Arbeitssituation herauszureissen.
    Gleichzeitig ist die Erinnerung an den gerade abgeschlossenen Arbeitsprozess noch präsent, was eine präzisere Einschätzung
    ermöglicht.

  \item \textbf{Nach Meetings:} Da Meetings selbst bereits eine Unterbrechung des regulären Arbeitsflusses darstellen, verursacht
    die Erfassung der Zufriedenheit direkt im Anschluss keine zusätzliche Störung. In diesem Moment können die Teilnehmenden
    ausserdem am besten einschätzen, wie produktiv, zielführend und angenehm die Besprechung empfunden wurde, wodurch die Angaben
    besonders aussagekräftig sind.
\end{itemize}

\subsection{Erfassung der Meetingqualität}

Ein weiterer Bestandteil der Erweiterung ist die Integration einer Funktion zur Bewertung von Meetings. Ziel ist es, nach
Abschluss einer Besprechung zeitnah eine kurze Rückmeldung zur wahrgenommenen Produktivität und Relevanz zu erfassen. Es wird
angenommen, dass durch die systematische Erhebung solcher Rückmeldungen der Einfluss einzelner Meetings auf die
Arbeitszufriedenheit sichtbar wird und so eine fundierte Grundlage für Verbesserungen der Meetingkultur entsteht.

Zur Bewertung werden sieben Kriterien herangezogen, die in Anlehnung an Best Practices und gängige Meeting-Umfrageinstrumente
ausgewählt wurden:

\begin{itemize}
  \item \textbf{Zielklarheit und Relevanz:} Bewertet, ob die Ziele des Meetings klar formuliert und inhaltlich relevant waren. 
    Eine präzise Agenda und eindeutige Zieldefinition gelten als zentrale Erfolgsfaktoren.

  \item \textbf{Inhaltliche Tiefe und Verständlichkeit:} Misst, ob die behandelten Themen angemessen tief und gleichzeitig 
    verständlich präsentiert wurden. Unnötige Detailfülle wird vermieden.

  \item \textbf{Zeitmanagement:} Prüft, ob die geplante Dauer eingehalten und das Meeting pünktlich begonnen wurde. Studien 
    zeigen, dass bewusst kürzere Meetings (z. B. 25 statt 30 Minuten) die Effizienz steigern können.

  \item \textbf{Moderation und Beteiligung:} Erfasst die Qualität der Moderation und die aktive Einbindung der Teilnehmenden. 
    Eine hohe Beteiligungsquote gilt als Indikator für Engagement und Interaktivität.

  \item \textbf{Ergebnisorientierung:} Bewertet, ob aus dem Meeting konkrete Entscheidungen oder Aufgaben resultierten. Die Anzahl 
    abgeschlossener Aktionspunkte kann als Kennzahl dienen.

  \item \textbf{Allgemeine Zufriedenheit:} Ermittelt das subjektive Gesamturteil der Teilnehmenden, beispielsweise auf einer 
    Skala von 1 bis 10.

  \item \textbf{Meetingdauer:} Vergleicht die geplante mit der tatsächlichen Dauer und bewertet, ob diese angemessen war.
\end{itemize}

\subsection{Verknüpfung von Zufriedenheitswerten mit Kontext- und Gesundheitsdaten}

Ein weiterer Bestandteil der Erweiterung ist die Verknüpfung der erfassten Zufriedenheitswerte mit zusätzlichen, für die
Interpretation relevanten Einflussfaktoren. Ziel ist es, die Aussagekraft der Daten zu erhöhen und ein umfassenderes Bild der
Zusammenhänge zwischen Arbeitsbedingungen und Arbeitszufriedenheit zu erhalten. Hierzu werden sowohl Daten aus dem Arbeitskontext
als auch ausgewählte Gesundheitsdaten berücksichtigt.

Unter Arbeitskontextdaten fallen Commitinformationen, wie Zeitpunkt und die Commit Message, welche eine kurze Zusammenfassung der
Codeänderungen darstellt. Ausserdem werden zusätzliche Informationen zu Meetings erfasst, beispielsweise der Name des Meetings oder
die Anzahl teilnehmender Personen.

Die Ergänzung der Zufriedenheitswerte um Gesundheitsdaten soll das subjektive Stimmungsbild erweitern und objektive Indikatoren für
Belastung, Erholung und allgemeines Wohlbefinden einbeziehen. Während subjektive Angaben wertvolle Einblicke in die aktuelle
emotionale Lage geben, erfassen physiologische Metriken Veränderungen, die den Betroffenen nicht immer unmittelbar bewusst sind.
Die Kombination beider Perspektiven ermöglicht eine fundiertere Analyse möglicher Einflussfaktoren auf die Arbeitszufriedenheit
von Entwicklerinnen und Entwicklern.

Für die Auswahl der relevanten Gesundheitsmetriken wurden drei Kriterien herangezogen:

\begin{itemize}
  \item \textbf{Wissenschaftlich belegter Zusammenhang} mit Arbeitszufriedenheit oder Leistungsfähigkeit.
  \item \textbf{Technische Messbarkeit} mittels gängiger Wearables bzw. Smartphone-Sensoren.
  \item \textbf{Integrationsfähigkeit} über etablierte Schnittstellen wie Apple Health Kit oder die Garmin Health API.
\end{itemize}

\todo{Referenz einfügen}
Die im Kapitel~\ref{Gesundheit} beschriebenen Gesundheitsmetriken erfüllen diese Kriterien und umfassen:

\begin{itemize}
\item \textbf{Schlafdauer}
\item \textbf{Ruheherzfrequenz (RHR)}
\item \textbf{Stress} (gemessen über die Herzratenvariabilität, HRV)
\item \textbf{Aktivitätsminuten und Schritte}
\end{itemize}

\subsection{Automatisierte Interpretation und Handlungsempfehlungen durch den Yappi Coach}

Als letzer Bestandteil der Erweiterung wertet der Yappi Coach die erfassten Daten automatisiert aus. Ziel ist es, nicht nur 
Rohdaten bereitzustellen, sondern diese in einen interpretierbaren Kontext zu setzen und daraus konkrete, umsetzbare 
Handlungsempfehlungen abzuleiten.

Der Yappi Coach kombiniert Zufriedenheitswerte, Kontextinformationen und ausgewählte Gesundheitsmetriken zu einem Gesamtbild der
aktuellen Arbeitssituation. Dabei werden wiederkehrende Muster, Auffälligkeiten und Abweichungen von individuellen Referenzwerten
identifiziert. Durch diese ganzheitliche Betrachtung können potenzielle Ursachen für Veränderungen in der Arbeitszufriedenheit
erkannt werden.

Ein wesentliches Ziel ist es, die Entwicklerinnen und Entwickler direkt bei der Verbesserung ihrer Arbeitssituation zu 
unterstützen. Statt lediglich Zahlen und Diagramme anzuzeigen, formuliert der Yappi Coach daraus spezifische Empfehlungen, die 
auf die jeweilige Personzugeschnitten sind. Beispiele hierfür sind:

\begin{itemize}
    \item Vorschläge zur Anpassung der Meetingfrequenz oder -dauer, wenn wiederholt ein negativer Zusammenhang zwischen
      Meetingbelastung und Zufriedenheit erkennbar ist.
    \item Hinweise auf mögliche Überlastung, wenn Gesundheitsmetriken wie Ruheherzfrequenz oder Schlafdauer über einen längeren
      Zeitraum ungünstige Werte aufweisen.
    \item Empfehlungen zur Optimierung des Arbeitsrhythmus, wenn bestimmte Tageszeiten oder Arbeitsphasen wiederholt mit höheren
      Zufriedenheitswerten korrelieren.
\end{itemize}

Der Yappi Coach soll somit die Brücke zwischen Datenerhebung und konkreter Verbesserung schlagen. Durch die kontinuierliche, 
automatisierte Analyse der Daten wird eine proaktive Unterstützung möglich, die nicht nur reaktiv auf bestehende Probleme eingeht,
sondern auch frühzeitig präventive Massnahmen anstösst.

\section{Vision, Mission und Leitprinzipien}

Die Weiterentwicklung von Yappi folgt einer klaren strategischen Ausrichtung, die sowohl die langfristigen Ziele (Vision) als auch 
den konkreten Handlungsauftrag (Mission) definiert. Dieses Kapitel beschreibt, welches Zukunftsbild mit Yappi angestrebt wird und
nach welchen Grundsätzen die Umsetzung erfolgt.

\subsection{Vision: Yappi als zentrale Plattform für Entwicklerzufriedenheit}

Yappi soll sich zu einer zentralen, intelligenten Plattform entwickeln, die es Entwicklerinnen, Entwicklern und Teams ermöglicht,
die Arbeitszufriedenheit kontinuierlich, präzise und im Kontext zu erfassen, zu verstehen und gezielt zu verbessern. Ziel ist ein
Arbeitsumfeld, in dem Zufriedenheit als gleichwertiger Faktor neben Produktivität und Codequalität betrachtet wird und in dem
datenbasierte Erkenntnisse zu einer nachhaltig gesunden und motivierenden Arbeitskultur beitragen.

\subsection{Mission: Nahtlose Erfassung, Analyse und Optimierung im Arbeitsalltag}

Die Mission von Yappi besteht darin, durch nahtlose Integration in den Arbeitsalltag verlässliche und kontextbezogene
Zufriedenheitsdaten zu erfassen, diese mit relevanten Kontext- und Gesundheitsmetriken zu verknüpfen und mithilfe intelligenter
Analysen konkrete Handlungsempfehlungen bereitzustellen. Dabei wird besonderer Wert darauf gelegt, den Erfassungs- und
Auswertungsprozess so zu gestalten, dass er minimal störend und auf den individuellen wie auch den Nutzen für Teams ausgerichtet
ist.

\subsection{Leitprinzipien: Werte und Ausrichtung der Plattform}

Die folgenden Leitsätze definieren den Rahmen, an dem sich die Weiterentwicklung von Yappi orientiert:

\begin{enumerate}
  \item \textbf{Ganzheitlichkeit vor Fragmentierung:} Zufriedenheitswerte werden immer im Kontext weiterer relevanter Faktoren 
    betrachtet.
  \item \textbf{Nahtlose Integration:} Die Nutzung soll den Arbeitsfluss nicht unterbrechen, sondern sich harmonisch einfügen.
  \item \textbf{Datenbasiert und handlungsorientiert:} Die Analysen zielen stets auf konkrete, umsetzbare Handlungsempfehlungen.
\end{enumerate}

\section{Value Proposition}

Dieses Kapitel beschreibt den konkreten Mehrwert, den die Erweiterung von Yappi für verschiedene Zielgruppen bietet. 
Dabei wird zwischen Einzelpersonen und Teams unterschieden, um die Vorteile jeweils im passenden Anwendungskontext darzustellen.

\subsection{Mehrwert für Einzelpersonen – Persönliche Optimierung der Arbeitszufriedenheit}

Die Erweiterung von Yappi ermöglicht es Entwicklerinnen und Entwicklern, ihre Arbeitszufriedenheit regelmässig und ohne hohen 
Mehraufwand zu erfassen. Durch die automatisierte Aufforderung zu geeigneten Zeitpunkten, die Anreicherung der Daten mit Kontext- 
und Gesundheitsinformationen sowie die Interpretation durch den Yappi Coach, erhalten Einzelpersonen fundierte und individuell 
relevante Rückmeldungen.  

Die wichtigsten Vorteile für Einzelpersonen sind:
\begin{itemize}
    \item \textbf{Regelmässige, unkomplizierte Erfassung} von Zufriedenheitsdaten ohne manuelles Initiieren.
    \item \textbf{Individuell zugeschnittene Empfehlungen} zur Verbesserung der eigenen Arbeitssituation.
    \item \textbf{Frühzeitige Erkennung von Belastungstendenzen} durch Kombination subjektiver und objektiver Daten.
    \item \textbf{Bessere Selbstreflexion} durch kontinuierliche und leicht zugängliche Verlaufsauswertungen.
\end{itemize}

\subsection{Mehrwert für Teams – Gemeinsame Verbesserung von Zusammenarbeit und Prozessen}

Auf Teamebene bietet die Erweiterung die Möglichkeit, kollektive Muster in der Arbeitszufriedenheit zu erkennen. Die gesammelten
Daten können aufzeigen, wie sich bestimmte Arbeitsweisen, Meetingstrukturen oder Prozessänderungen auf die Gesamtzufriedenheit im
Team auswirken.  

Die wichtigsten Vorteile für Teams sind:
\begin{itemize}
    \item \textbf{Transparenz über gemeinsame Herausforderungen} durch aggregierte Zufriedenheitsdaten.
    \item \textbf{Fundierte Entscheidungsgrundlage} für Prozess- und Arbeitsablaufoptimierungen.
    \item \textbf{Verbesserung der Meetingkultur} durch systematische Erfassung der Meetingqualität.
    \item \textbf{Stärkung der Zusammenarbeit} durch gezielte, datenbasierte Anpassungen.
\end{itemize}

\section{Produktziele}

Die im vorangehenden Kapitel beschriebenen Erweiterungsideen bilden die Grundlage für die folgenden Produktziele. Sie fassen die
wesentlichen Eigenschaften zusammen, die Yappi im Rahmen der geplanten Weiterentwicklung erfüllen soll, um die identifizierten
Problemstellungen wirksam zu adressieren.

\subsection{Z-1: Automatisierte Aufforderung zur Zufriedenheitserfassung}

\subsubsection{Ziel:}

Yappi soll Entwicklerinnen und Entwickler automatisch zu geeigneten Zeitpunkten auffordern, Zufriedenheitsdaten zu erfassen, ohne
den Arbeitsfluss zu stören. Dadurch wird eine kontinuierliche und vollständige Datengrundlage geschaffen, die eine präzisere
Analyse der Zufriedenheit ermöglicht.

\subsubsection{Beschreibung:}

Die automatisierte Aufforderung erfolgt kontextbezogen, beispielsweise direkt nach einem Commit oder unmittelbar nach einem
Meeting. Dies gewährleistet, dass die Erfassung in Momenten erfolgt, in denen der Arbeitsfluss ohnehin kurz unterbrochen ist, und
reduziert so den wahrgenommenen Mehraufwand für die Nutzerinnen und Nutzer.

\subsubsection{Bezug zur Forschungsfrage:}

Bezieht sich auf Forschungsfrage A: Durch den Einsatz von Technologien und Schnittstellen wie IDE-Plugins, Browsererweiterungen
oder Integrationen in Tools wie Microsoft Teams und Outlook wird ein reibungsloses und einfaches Erfassen von Zufriedenheitsdaten
ermöglicht.

\subsection{Z-2: Erfassung der Meetingqualität}

\subsubsection{Ziel:}

Yappi soll nach relevanten Meetings zeitnah eine kurze Rückmeldung zur wahrgenommenen Produktivität, Relevanz und Belastung
einholen, um den Einfluss einzelner Besprechungen auf die Arbeitszufriedenheit sichtbar zu machen.

\subsubsection{Beschreibung:}

Die Bewertung erfolgt anhand vordefinierter Kriterien wie Zielklarheit, Moderationsqualität, Zeitmanagement und
Ergebnisorientierung. Die Integration in gängige Meeting-Tools stellt sicher, dass die Abfrage ohne zusätzlichen manuellen Aufwand
in den Arbeitsalltag eingebettet wird.

\subsubsection{Bezug zur Forschungsfrage:}

Bezieht sich auf Forschungsfrage A: Durch die Integration in Meeting-Plattformen wie Microsoft Teams oder Outlook werden
bestehende Technologien genutzt, um die Erfassung im direkten Arbeitskontext zu automatisieren und den Erfassungsprozess für die
Nutzerinnen und Nutzer zu vereinfachen.

\subsection{Z-3: Verknüpfung mit Kontext- und Gesundheitsdaten}

\subsubsection{Ziel:}

Yappi soll Zufriedenheitswerte mit automatisch erfassten Kontextinformationen aus dem Arbeitsumfeld sowie ausgewählten
Gesundheitsdaten verknüpfen, um ein vollständigeres Bild der Einflussfaktoren auf die Arbeitszufriedenheit zu erhalten.

\subsubsection{Beschreibung:}

Neben Arbeitskontextdaten wie Commit-Zeitpunkt, Commit-Message und Meetinginformationen werden Gesundheitsmetriken wie
Schlafdauer, Ruheherzfrequenz, Stress (HRV) sowie Aktivitätsminuten und Schritte einbezogen. Diese Daten werden über etablierte
Schnittstellen wie Apple HealthKit oder die Garmin Health API automatisiert integriert.

\subsubsection{Bezug zur Forschungsfrage:}

Bezieht sich auf Forschungsfrage B: Die Anbindung an Gesundheitsdaten-APIs ermöglicht es, wissenschaftlich relevante und technisch
messbare Metriken in die Analyse der Entwicklerzufriedenheit einzubeziehen, um Zusammenhänge zwischen Wohlbefinden und
Zufriedenheit datenbasiert zu identifizieren.

\subsection{Z-4: Automatisierte Interpretation und Handlungsempfehlungen (Yappi Coach)}

\subsubsection{Ziel:}

Yappi soll die erfassten Zufriedenheits-, Kontext- und Gesundheitsdaten automatisiert analysieren und daraus konkrete, individuell
zugeschnittene Handlungsempfehlungen ableiten.

\subsubsection{Beschreibung:}

Der Yappi Coach erkennt Muster, Auffälligkeiten und Abweichungen von individuellen Referenzwerten, um frühzeitig potenzielle
Probleme zu identifizieren. Auf Basis dieser Analysen werden praxisnahe, datengestützte Empfehlungen gegeben, die sowohl
präventive Massnahmen als auch Optimierungen im Arbeitsalltag umfassen.

\subsubsection{Bezug zur Forschungsfrage:}

Bezieht sich auf Forschungsfrage C: Durch die Nutzung KI-gestützter Auswertungen können aus den kombinierten Daten gezielte,
evidenzbasierte Vorschläge zur Verbesserung der Arbeitsbedingungen und der Zufriedenheit abgeleitet werden.

\subsubsection{Anmerkung:}

Dieses Ziel wird im Rahmen der vorliegenden Arbeit nicht technisch umgesetzt, sondern ausschliesslich konzeptionell ausgearbeitet.
Die praktische Implementierung ist als Bestandteil eines Nachfolgeprojekts vorgesehen.

\printbibliography

\end{document}
