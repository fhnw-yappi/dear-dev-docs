\documentclass[12pt,a4paper]{report}

\usepackage[backend=biber, style=ieee]{biblatex}
\addbibresource{literature.bib}

\usepackage[utf8]{inputenc}
\usepackage[T1]{fontenc}
\usepackage[ngerman]{babel}
\usepackage{lmodern}
\usepackage{enumitem}
\usepackage{graphicx}
\usepackage{tabularx}
\usepackage{float}
\usepackage[onehalfspacing]{setspace}
\usepackage{microtype}
\usepackage{parskip}

\usepackage{xcolor}
\newcommand{\todo}[1]{\colorbox{red}{\textbf{TODO: #1}}\\}
\newcommand{\question}[1]{\colorbox{yellow}{\textbf{QUESTION: #1}}\\}
\newcommand{\xeno}[1]{\colorbox{pink}{\textbf{TODO XENO: #1}}\\}
\newcommand{\gideon}[1]{\colorbox{green}{\textbf{TODO GIDEON: #1}}\\}

\begin{document}

\tableofcontents
\newpage

\chapter{Konzeptlösung für die Erweiterungen an Yappi}

Die im vorhergehenden Kapitel beschriebenen Problemstellungen und die Analyse bestehender Lösungen verdeutlichen, dass aktuelle
Ansätze zur Erfassung der Entwicklerzufriedenheit häufig unvollständig sind. Sie erfassen den Einfluss einzelner Arbeitsereignisse
nicht systematisch und bieten nur eingeschränkte Möglichkeiten, die erhobenen Daten im individuellen Arbeitskontext zu
interpretieren. Zudem werden potenziell relevante Einflussfaktoren, wie das physische Wohlbefinden, nur selten berücksichtigt.

Mit den geplanten Erweiterungen soll Yappi genau diese Lücken schliessen und eine integrierte, praxisorientierte Lösung
bereitstellen, die Entwicklerinnen, Entwickler, Teams und Organisationen gleichermassen unterstützt. Ziel ist es, die
Entwicklerzufriedenheit umfassend zu erfassen und die gewonnenen Erkenntnisse in konkrete Handlungsempfehlungen zu überführen.

Die Erweiterung basiert auf folgenden Ansätzen:

\begin{itemize}

    \item Literaturrecherche zum Stand der Forschung im Bereich der Entwicklerzufriedenheit und deren Einfluss auf die
      Produktivität
    \item Analyse bestehender Lösungen zur Erfassung und Auswertung von Entwicklerzufriedenheit, Arbeitskontext und 
      Gesundheitsdaten im Umfeld der Softwareentwicklung
    \item Befragungen von Entwicklerinnen und Entwicklern in den Unternehmen der Autorinnen und Autoren
    \item Brainstorming-Sitzungen mit den Projektbetreuenden
\end{itemize}

In den folgenden Unterkapiteln wird die schrittweise Entwicklung der Konzeptlösung für die geplanten Erweiterungen von Yappi
detailliert beschrieben.

\section{Problemstellungen}

Die Erweiterung von Yappi verfolgt das Ziel, bestehende Schwächen in der Erfassung und Interpretation der Entwicklerzufriedenheit
zu beheben. Dazu wurden die relevanten Herausforderungen nach dem Standard des Rational Unified Process (RUP) strukturiert
beschrieben. Jede Problemstellung wird dabei in einem einheitlichen Schema dargestellt, um die Auswirkungen klar herauszuarbeiten
und eine fundierte Basis für die Definition der Projektziele zu schaffen.

Die folgenden Abschnitte fassen die drei zentralen Problemstellungen zusammen, die mit der Erweiterung adressiert werden sollen.

\subsection{Passive Erfassung der Entwicklerzufriedenheit}

In vielen aktuellen Umsetzungen erfolgt die Erfassung der Zufriedenheit rein passiv, d.\,h., Entwicklerinnen und Entwickler
müssen von sich aus aktiv eine Rückmeldung abgeben. Dieser Prozess ist stark von der Eigeninitiative abhängig und wird im
Arbeitsalltag häufig vergessen oder aufgeschoben. Dadurch entsteht eine unvollständige und unregelmässige Datengrundlage, die den
tatsächlichen Verlauf der Zufriedenheit nur eingeschränkt widerspiegelt. Die vollständige Problemstellung ist in 
Tabelle~\ref{tab:rup-passive} dargestellt.

\begin{table}[H]
  \caption{Problemstellung nach RUP: Passive erfassung der Entwicklerzufriedenheit}
  \label{tab:rup-passive}
  \centering
  \begin{tabularx}{\textwidth}{>{\raggedright\arraybackslash}p{0.25\textwidth}|X}
    \hline
    Das Problem & der passiven Erfassung der Entwicklerzufriedenheit,
    bei der ein Entwickler von sich aus eine Rückmeldung abgeben muss \\ \hline

    betrifft & Entwicklerinnen und Entwickler. \\ \hline

    Die Auswirkung dieses Problems & ist eine geringe Erfassungsrate der Zufriedenheitsdaten,
    da Feedback oft vergessen oder aufgeschoben wird. \\ \hline

    Eine erfolgreiche Lösung & fordert Entwicklerinnen und Entwickler zu relevanten
    und regelmässigen Zeitpunkten automatisiert dazu auf, Zufriedenheitsdaten zu erfassen.
    Dabei darf kein grosser Mehraufwand entstehen, um zur Abgabe zu motivieren. \\ \hline
  \end{tabularx}
\end{table}

\subsection{Schwierigkeiten beim Ableiten von Schlussfolgerungen}

Die Erhebung reiner Zufriedenheitswerte ohne weiterführende Informationen erschwert es, deren Ursachen oder Auslöser zu verstehen.  
Ohne zusätzlichen Kontext ist es schwierig, Muster zu erkennen oder Zusammenhänge zwischen bestimmten Arbeitsbedingungen und der
Zufriedenheit herzustellen. Dies führt dazu, dass Massnahmen zur Verbesserung oft auf Annahmen basieren und nicht ausreichend
datenbasiert sind. Die vollständige Problemstellung ist in Tabelle~\ref{tab:rup-conclusions} dargestellt.

\begin{table}[H]
  \caption{Problemstellung nach RUP: Schwierigkeiten beim ableiten von Schlussfolgerungen aus Zufriedenheitsdaten}
  \label{tab:rup-conclusions}
  \centering
  \begin{tabularx}{\textwidth}{>{\raggedright\arraybackslash}p{0.25\textwidth}|X}
    \hline
    Das Problem & dass es schwierig ist, aus den erfassten Zufriedenheitsdaten klare Schlussfolgerungen abzuleiten \\ \hline

    betrifft & Scrum Master, Product Owner, Projektverantwortliche, Entwicklerinnen und Entwickler. \\ \hline

    Die Auswirkung dieses Problems & ist, dass unklar bleibt, welche Faktoren oder Ereignisse zu den gemessenen Ergebnissen 
    geführt haben, wodurch gezielte Massnahmen zur Verbesserung erschwert werden. \\ \hline

    Eine erfolgreiche Lösung & ergänzt die Zufriedenheitsmessungen um automatisch erfasste Kontextinformationen aus dem 
    Arbeitsumfeld sowie ausgewählte Gesundheitsdaten, um die Ergebnisse besser einordnen und deren Ursachen gezielter 
    identifizieren zu können. Zusätzlich leitet sie konkrete Handlungsempfehlungen aus diesen Daten ab. \\ \hline
  \end{tabularx}
\end{table}

\subsection{Einfluss einzelner Meetings}

Meetings nehmen einen wesentlichen Teil der Arbeitszeit von Entwicklerinnen und Entwicklern ein. Werden sie als unproduktiv oder
belastend wahrgenommen, kann dies die Arbeitszufriedenheit deutlich beeinträchtigen. Derzeit wird jedoch der Einfluss einzelner
Besprechungen auf die Zufriedenheit nicht systematisch erfasst, was eine gezielte Verbesserung der Meetingkultur erschwert. Die
vollständige Problemstellung ist in Tabelle~\ref{tab:rup-meetings} dargestellt.

\begin{table}[H]
  \caption{Problemstellung nach RUP: Einfluss auf die Zufriedenheit einzelner Meetings wird nicht erfasst}
  \label{tab:rup-meetings}
  \centering
  \begin{tabularx}{\textwidth}{>{\raggedright\arraybackslash}p{0.25\textwidth}|X}
    \hline
    Das Problem & dass der Einfluss einzelner Meetings auf die Zufriedenheit nicht erfasst wird \\ \hline

    betrifft & Scrum Master, Product Owner, Entwicklerinnen und Entwickler. \\ \hline

    Die Auswirkung dieses Problems & ist, dass unproduktive oder belastende Besprechungen schwer identifiziert werden können und
    deren Auswirkungen auf die tägliche Arbeit unbekannt bleiben. \\ \hline

    Eine erfolgreiche Lösung & erfasst nach relevanten Besprechungen zeitnah die wahrgenommene Produktivität und Belastung, um
    den Einfluss einzelner Meetings auf die Arbeitszufriedenheit sichtbar zu machen. \\ \hline
  \end{tabularx}
\end{table}

\section{Grundidee}
\section{Vision und Mission}
\section{Value Proposition}
\section{Produktziele}
\todo{Anmerkung }
\section{Zielgruppe}

\printbibliography

\end{document}
