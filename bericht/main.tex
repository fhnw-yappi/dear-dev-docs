\documentclass[12pt,a4paper]{report}

\usepackage[utf8]{inputenc}
\usepackage[T1]{fontenc}
\usepackage[ngerman]{babel}
\usepackage{lmodern}
\usepackage{enumitem}

% Hilfscommand für TODO""
\usepackage{xcolor}
\newcommand{\todo}[1]{\colorbox{red}{\textbf{TODO: #1}}}

\begin{document}

\begin{titlepage}
    \centering
    {\huge \textbf{Yappi - Developer Happiness} \par}
    {\large IP5 Project \par}
    \vspace{0.5cm}
    {Windisch, August 2025 \par}
    \vspace{0.5cm}

    \begin{tabular}{@{}ll@{}}
        \textbf{Studenten:} & Xeno Isenegger, Gideon Monterosa \\
        \textbf{Fachbetreuer:} & Norbert Seyff, Nitish Patkar
    \end{tabular}

    \vfill
    {Fachhochschule Nordwestschweiz, Hochschule für Informatik \par}
\end{titlepage}

\chapter*{Abstract}
\newpage

\tableofcontents
\newpage

\listoffigures
\newpage

\listoftables
\newpage

\chapter{Einleitung}

\section{Hintergrund und Motivation}
\todo{evt Ausgangslage}

\todo{Text aus dem Project Agreement noch zu überarbeiten}

Die Zufriedenheit von Entwicklerinnen und Entwicklern wird, wenn überhaupt, meist nur anhand der
Menge ihrer geleisteten Arbeit gemessen. Dabei entstehen zwangsläufig Defizite, und ein halbjährli-
ches Mitarbeitergespräch erweist sich oft als wenig wirksame Massnahme zur Problemlösung.

Dieses Projekt baut auf einer bestehenden Arbeit auf, in der eine Plattform zur Erfassung der Ent-
wicklerzufriedenheit entwickelt wurde. Die Webapplikation Yappi ermöglicht es Entwicklerinnen und
Entwicklern, ihre Zufriedenheit mit ihrer Arbeit und ihrer aktuellen Situation fortlaufend zu bewerten.
Yappi erfasst emotionale Faktoren wie Happiness sowie weitere Zufriedenheitsindikatoren. Zusätz-
lich können spezifische Aufgaben und Arbeitstypen individuell bewertet werden. Die erhobenen Da-
ten werden anonym auf Teamebene analysiert, um ein fundiertes Verständnis für die Stimmung in-
nerhalb der Teams zu gewinnen.

Entwicklerinnen und Entwickler haben die Möglichkeit, ihre Zufriedenheit für verschiedene Teams zu
erfassen, wodurch gezielte Analysen ermöglicht werden. Unternehmen erhalten dadurch wertvolle
Einblicke, um das Arbeitsumfeld gezielt zu verbessern.

Unser Projekt baut auf Yappi auf und zielt darauf ab, die Erfassung der Zufriedenheit weiter zu opti-
mieren. Es wird untersucht, wie die Daten noch präziser erfasst und ausgewertet werden können,
um langfristige Verbesserungen zu unterstützen. Diese Arbeit dient als Grundlage für ein weiterfüh-
rendes Forschungsprojekt, das sich vertieft mit der Entwicklerzufriedenheit auseinandersetzt und
zusätzliche Erkenntnisse gewinnen soll.

\section{Ziele und Vision}

\todo{Text aus dem Project Agreement noch zu überarbeiten}

Yappi wird zu einer umfassenden Plattform weiterentwickelt, die nicht nur die Zufriedenheit misst,
sondern sich nahtlos in den Arbeitsalltag integriert und wertvolle Handlungsempfehlungen liefert.
Dazu werden folgende Kernaspekte umgesetzt:

\textbf{Produktivitätsfaktoren identifizieren}

Durch eine tiefere Analyse von Zufriedenheitsindikatoren sollen zentrale Faktoren ermittelt werden,
die sich positiv oder negativ auf die Produktivität und das Wohlbefinden von Entwicklerinnen und
Entwicklern auswirken. Diese Erkenntnisse werden genutzt, um Vorschläge zu Verbesserungsmass-
nahmen abzuleiten.

\textbf{Integration in den Arbeitsprozess}

Yappi soll sich direkt in bestehende Arbeitsabläufe einfügen, um die Erfassung der Zufriedenheit
möglichst intuitiv und effizient zu gestalten. Dies kann durch verschiedene Schnittstellen und Erwei-
terungen erfolgen, die eine nahtlose Interaktion ermöglichen.

\textbf{Erweiterung um kontextbezogene Daten}

Um ein umfassenderes Bild der Arbeitszufriedenheit zu erhalten, können weitere Einflussfaktoren
berücksichtigt werden. Dazu gehören beispielsweise arbeitsbezogene Rahmenbedingungen oder
individuelle Gesundheits- und Belastungsindikatoren. Diese Daten sollen helfen, ein besseres Ver-
ständnis für langfristige Trends und Zusammenhänge zu entwickeln.

\textbf{Intelligente Analyse und Handlungsempfehlungen}

Durch die Integration von AI schnittstellen können gezielte Analysen erstellt und individualisierte
Empfehlungen abgeleitet werden. Dies kann sowohl auf individueller als auch auf Teamebene erfol-
gen, um nachhaltige Verbesserungen im Arbeitsumfeld zu fördern.

\textbf{Fazit}

Mit diesen Erweiterungen wird Yappi zu einem essenziellen Bestandteil des Entwickleralltags. Es
bietet nicht nur eine präzisere Erfassung der Zufriedenheit, sondern liefert auch wertvolle Einblicke
und Handlungsempfehlungen, um die Arbeitsbedingungen nachhaltig zu verbessern. Unternehmen
erhalten fundierte Analysen und können gezielt Massnahmen ergreifen, um eine motivierte und pro-
duktive Entwicklergemeinschaft zu fördern.

\section{Fragestellungen}

\todo{Text aus dem Project Agreement noch zu überarbeiten}

\begin{enumerate}[label=\Alph*.] % A, B, C ...
    \item Durch welche Technologien und Schnittstellen kann Yappi erweitert werden, 
    um ein reibungsloses und einfaches Erfassen von Zufriedenheitsdaten zu ermöglichen?
    \begin{enumerate}[label=\alph*.] % a, b, c ...
        \item Entwicklung von Entwickler-Tool-Plugins, die nahtlos in bestehende Arbeitsumgebungen 
        integriert werden können, um die Nutzung von Yappi angenehmer und effizienter zu gestalten. 
        Diese Plugins sollen Entwicklern ermöglichen, direkt in ihrer bevorzugten Umgebung Feedback 
        zu erfassen, ohne den Arbeitsfluss zu unterbrechen. Integration von Yappi in verschiedene 
        Plattformen und Tools wie Webbrowser, IntelliJ, Microsoft Teams und Outlook.
    \end{enumerate}
    \item Wie können Gesundheitsdaten in die Auswertung der Entwicklerzufriedenheit einfließen?
    \begin{enumerate}[label=\alph*.]
        \item Direkte Anbindung der Gesundheitsdaten-API, um relevante Gesundheitsmetriken wie 
        Herzfrequenz, Schlafqualität oder Stresslevel automatisch in die Analyse der 
        Entwicklerzufriedenheit zu integrieren. Dies ermöglicht eine genauere Einschätzung des 
        Wohlbefindens und potenzieller Belastungsfaktoren.
    \end{enumerate}
    \item Wie kann Yappi Teams und Entwickler dabei unterstützen, aus den erfassten 
    Zufriedenheitsdaten Handlungsempfehlungen abzuleiten, um die Zufriedenheit und Produktivität 
    von Entwicklern zu erhöhen?
    \begin{enumerate}[label=\alph*.]
        \item Entwicklung eines Yappi Coach, der anhand einer detaillierten Analyse der erfassten 
        Daten gezielte Tipps zur Verbesserung der Arbeitsweise gibt. Beispielsweise könnte der Coach 
        darauf hinweisen, dass Meetings nicht länger als 1,5 Stunden dauern sollten, da längere 
        Sitzungen die Zufriedenheit und Konzentration der Entwickler negativ beeinflussen können.
        \item Integration von KI-gestützten Diensten, die auf Basis der gesammelten Gesundheitsdaten 
        sowie Zufriedenheits- und Produktivitätsmetriken individuelle Maßnahmen vorschlagen. Diese 
        KI-gestützten Empfehlungen können Teams dabei helfen, gezielt Optimierungen vorzunehmen, um 
        die Arbeitsbedingungen und die Effizienz der Entwickler nachhaltig zu verbessern.
    \end{enumerate}
\end{enumerate}

\chapter{Hintergrund}
\chapter{State of the Art}

\chapter{Methoden}
\section{Projektmethodik}
\section{Prototypen}
\section{Proof of Concepts}

\chapter{Konzeptentwurf}
\section{Zugriffskontrolle über API Keys}
\section{Companion Apps}
\subsection{Integration in die Entwicklungsumgebung}
\subsection{Kalender Integration}
\subsection{Integratin von Gesundheitsdaten}
\section{Yappi Coach}
\section{Konzeptevaluation}

\chapter{Implementierung}
\section{Zugriffskontrolle über API Keys}
\section{Companion Apps}
\subsection{IntelliJ IDEA Companion}
\subsection{Calendar Companion}
\subsection{Health Companion}

\chapter{Evaluation}

\chapter{Quellen}


\end{document}
