\documentclass[12pt,a4paper]{report}

\usepackage[backend=biber, style=ieee]{biblatex}
\addbibresource{literature.bib}

\usepackage[utf8]{inputenc}
\usepackage[T1]{fontenc}
\usepackage[ngerman]{babel}
\usepackage{lmodern}
\usepackage{enumitem}

\usepackage{xcolor}
\newcommand{\todo}[1]{\colorbox{red}{\textbf{TODO: #1}}\\}
\newcommand{\question}[1]{\colorbox{yellow}{\textbf{QUESTION: #1}}\\}
\newcommand{\xeno}[1]{\colorbox{pink}{\textbf{TODO XENO: #1}}\\}
\newcommand{\gideon}[1]{\colorbox{green}{\textbf{TODO GIDEON: #1}}\\}

\begin{document}

\begin{titlepage}
  \centering
  {\huge \textbf{Yappi - Developer Happiness} \par}
  {\large IP5 Project \par}
  \vspace{0.5cm}
  {Windisch, August 2025 \par}
  \vspace{0.5cm}

  \begin{tabular}{@{}ll@{}}
    \textbf{Studenten:}    & Xeno Isenegger, Gideon Monterosa \\
    \textbf{Fachbetreuer:} & Norbert Seyff, Nitish Patkar
  \end{tabular}

  \vfill
  {Fachhochschule Nordwestschweiz, Hochschule für Informatik \par}
\end{titlepage}

\chapter*{Abstract}
\newpage

\tableofcontents
\newpage

\listoffigures
\newpage

\listoftables
\newpage

\chapter{Einleitung}

\section{Hintergrund und Motivation}
\todo{evt Ausgangslage}

\todo{aus Project Agreement: zu überarbeiten}

Die Zufriedenheit von Entwicklerinnen und Entwicklern wird, wenn überhaupt, meist nur anhand der
Menge ihrer geleisteten Arbeit gemessen. Dabei entstehen zwangsläufig Defizite, und ein halbjährli-
ches Mitarbeitergespräch erweist sich oft als wenig wirksame Massnahme zur Problemlösung.

Dieses Projekt baut auf einer bestehenden Arbeit auf, in der eine Plattform zur Erfassung der Ent-
wicklerzufriedenheit entwickelt wurde. Die Webapplikation Yappi ermöglicht es Entwicklerinnen und
Entwicklern, ihre Zufriedenheit mit ihrer Arbeit und ihrer aktuellen Situation fortlaufend zu bewerten.
Yappi erfasst emotionale Faktoren wie Happiness sowie weitere Zufriedenheitsindikatoren. Zusätz-
lich können spezifische Aufgaben und Arbeitstypen individuell bewertet werden. Die erhobenen Da-
ten werden anonym auf Teamebene analysiert, um ein fundiertes Verständnis für die Stimmung in-
nerhalb der Teams zu gewinnen.

Entwicklerinnen und Entwickler haben die Möglichkeit, ihre Zufriedenheit für verschiedene Teams zu
erfassen, wodurch gezielte Analysen ermöglicht werden. Unternehmen erhalten dadurch wertvolle
Einblicke, um das Arbeitsumfeld gezielt zu verbessern.

Unser Projekt baut auf Yappi auf und zielt darauf ab, die Erfassung der Zufriedenheit weiter zu opti-
mieren. Es wird untersucht, wie die Daten noch präziser erfasst und ausgewertet werden können,
um langfristige Verbesserungen zu unterstützen. Diese Arbeit dient als Grundlage für ein weiterfüh-
rendes Forschungsprojekt, das sich vertieft mit der Entwicklerzufriedenheit auseinandersetzt und
zusätzliche Erkenntnisse gewinnen soll.

\section{Ziele und Vision}

\todo{Text aus dem Project Agreement noch zu überarbeiten}

Yappi wird zu einer umfassenden Plattform weiterentwickelt, die nicht nur die Zufriedenheit misst,
sondern sich nahtlos in den Arbeitsalltag integriert und wertvolle Handlungsempfehlungen liefert.
Dazu werden folgende Kernaspekte umgesetzt:

\textbf{Produktivitätsfaktoren identifizieren}

Durch eine tiefere Analyse von Zufriedenheitsindikatoren sollen zentrale Faktoren ermittelt werden,
die sich positiv oder negativ auf die Produktivität und das Wohlbefinden von Entwicklerinnen und
Entwicklern auswirken. Diese Erkenntnisse werden genutzt, um Vorschläge zu Verbesserungsmass-
nahmen abzuleiten.

\textbf{Integration in den Arbeitsprozess}

Yappi soll sich direkt in bestehende Arbeitsabläufe einfügen, um die Erfassung der Zufriedenheit
möglichst intuitiv und effizient zu gestalten. Dies kann durch verschiedene Schnittstellen und Erwei-
terungen erfolgen, die eine nahtlose Interaktion ermöglichen.

\textbf{Erweiterung um kontextbezogene Daten}

Um ein umfassenderes Bild der Arbeitszufriedenheit zu erhalten, können weitere Einflussfaktoren
berücksichtigt werden. Dazu gehören beispielsweise arbeitsbezogene Rahmenbedingungen oder
individuelle Gesundheits- und Belastungsindikatoren. Diese Daten sollen helfen, ein besseres Ver-
ständnis für langfristige Trends und Zusammenhänge zu entwickeln.

\textbf{Intelligente Analyse und Handlungsempfehlungen}

Durch die Integration von AI schnittstellen können gezielte Analysen erstellt und individualisierte
Empfehlungen abgeleitet werden. Dies kann sowohl auf individueller als auch auf Teamebene erfol-
gen, um nachhaltige Verbesserungen im Arbeitsumfeld zu fördern.

\textbf{Fazit}

Mit diesen Erweiterungen wird Yappi zu einem essenziellen Bestandteil des Entwickleralltags. Es
bietet nicht nur eine präzisere Erfassung der Zufriedenheit, sondern liefert auch wertvolle Einblicke
und Handlungsempfehlungen, um die Arbeitsbedingungen nachhaltig zu verbessern. Unternehmen
erhalten fundierte Analysen und können gezielt Massnahmen ergreifen, um eine motivierte und pro-
duktive Entwicklergemeinschaft zu fördern.

\section{Fragestellungen}

\todo{Text aus dem Project Agreement noch zu überarbeiten}

\begin{enumerate}[label=\Alph*.]
  \item Durch welche Technologien und Schnittstellen kann Yappi erweitert werden,
        um ein reibungsloses und einfaches Erfassen von Zufriedenheitsdaten zu ermöglichen?
        \begin{enumerate}[label=\alph*.]
          \item Entwicklung von Entwickler-Tool-Plugins, die nahtlos in bestehende Arbeitsumgebungen
                integriert werden können, um die Nutzung von Yappi angenehmer und effizienter zu gestalten.
                Diese Plugins sollen Entwicklern ermöglichen, direkt in ihrer bevorzugten Umgebung Feedback
                zu erfassen, ohne den Arbeitsfluss zu unterbrechen. Integration von Yappi in verschiedene
                Plattformen und Tools wie Webbrowser, IntelliJ, Microsoft Teams und Outlook.
        \end{enumerate}
  \item Wie können Gesundheitsdaten in die Auswertung der Entwicklerzufriedenheit einfliessen?
        \begin{enumerate}[label=\alph*.]
          \item Direkte Anbindung der Gesundheitsdaten-API, um relevante Gesundheitsmetriken wie
                Herzfrequenz, Schlafqualität oder Stresslevel automatisch in die Analyse der
                Entwicklerzufriedenheit zu integrieren. Dies ermöglicht eine genauere Einschätzung des
                Wohlbefindens und potenzieller Belastungsfaktoren.
        \end{enumerate}
  \item Wie kann Yappi Teams und Entwickler dabei unterstützen, aus den erfassten
        Zufriedenheitsdaten Handlungsempfehlungen abzuleiten, um die Zufriedenheit und Produktivität
        von Entwicklern zu erhöhen?
        \begin{enumerate}[label=\alph*.]
          \item Entwicklung eines Yappi Coach, der anhand einer detaillierten Analyse der erfassten
                Daten gezielte Tipps zur Verbesserung der Arbeitsweise gibt. Beispielsweise könnte der Coach
                darauf hinweisen, dass Meetings nicht länger als 1,5 Stunden dauern sollten, da längere
                Sitzungen die Zufriedenheit und Konzentration der Entwickler negativ beeinflussen können.
          \item Integration von KI-gestützten Diensten, die auf Basis der gesammelten Gesundheitsdaten
                sowie Zufriedenheits- und Produktivitätsmetriken individuelle Massnahmen vorschlagen. Diese
                KI-gestützten Empfehlungen können Teams dabei helfen, gezielt Optimierungen vorzunehmen, um
                die Arbeitsbedingungen und die Effizienz der Entwickler nachhaltig zu verbessern.
        \end{enumerate}
\end{enumerate}

\chapter{Hintergrund}
\todo{unterkapitel für den Stand von Yappi vor dem Projekt}

\chapter{State of the Art}
\section{Definition von Entwiklerzufriedenheit}

Entwicklerzufriedenheit wird in der Literatur als Balance zwischen positiven und negativen Erlebnissen bei der
Arbeit definiert. Darunter versteht man eine Sequenz von Erfahrungen, bei der häufige positive Emotionen ein
hohes Glücksgefühl erzeugen und häufige negative Erfahrungen das Gegenteil bewirken \cite{sadowski_happiness_2019}.
Auch Industriequellen fassen Entwicklerzufriedenheit als subjektives Wohlbefinden in Bezug auf Arbeitsinhalte
und -umfeld auf, d.h. als Mass für Zufriedenheit, Freude oder innere Zufriedenheit bei der Arbeit \cite{zenhub_2022_nodate}.
Zufriedene Entwickler empfinden demnach mehr Arbeitsfreude und Inhaltlichkeit in ihrer Rolle, was eng mit
der Arbeitsmotivation und dem Engagement bei der Arbeit verknüpft ist \cite{franca_motivation_2020}.

\todo{evt braucht es hier eine citation oder fussnote für den originalen Begriff von flow}
\todo{Ich bin mir unsicher ob die citation in diesem Absatz sauber genug ist}
Eng verwendt mit der Zufriedenheit ist der Begriff \textbf{Flow}. In Anlehnung an Csikszentmihalyis Konzept beschreibt
Flow einen Zustand von völliger Vertiefung und hohen Fokus beim Programmieren. Flow tritt dann auf, wenn die Anforderungen
einer Aufgabe im Gleichgewicht mit dem Fähigkeiten des Entwicklers stehen, wodurch man in einen Zustand von intensiver 
Konzentration gelangt. Zufriedene Entwickler gelangen einfacher in einen anhaltenden Flow-Zustand. Unzufriedenheit
hingegen unterbricht diesen Flow, was zu Frustration führt und Schwierigkeiten führt, nach Unterbrechungen wieder in
eine Aufgabe zurückzufinden. Teilnehmer einer Untersuchung berichten, negative Erlebnisse reissen einen aus dem Flow
Zustand und machen es schwer, die Arbeit wieder aufzunehmen \cite{sadowski_happiness_2019}.

\todo{maybe absatz über burnout}

Motivation und Zufriedenheit hängen eng zusammen, sind aber konzeptionell unterscheidbar. Motivierte Entwickler sind
zeigen hohes Engagement und Fokus auf ihre Aufgaben, während Zufriedenheiteher durch allgemeines Wohlbefinden und gute
Laune charakerisiert ist. Faktoren wie Autonomie, Kompetenzerleben und Zugehörigkeitsgefühl steigern die intrinsische 
Motivation von Entwicklern, was sich positiv auf ihre Zufriedenheit auswirkt. Zufriedenheit ist zugleich das Ergebnis und
die Voraussetzung von Motivation, zufriedenere Entwickler weisen in der Regel eine höhere Antriebskraft auf, was wiederum ihre
Arbeitszufriedenheit weiter stärkt \cite{franca_motivation_2020}.

\section{Stand der Forschung und verwandte Arbeiten}
\section{Bestehende Lösungen und Wettbewerbsanalyse}

\todo{kleine einleitung}

\question{Braucht es hier quellen?}
\question{Sind das zu viele?}
\begin{description}
  \item[Officevibe] ist ein SaaS-Tool für wöchentliche Puls-Umfragen, das primär die allgemeine Mitarbeiter­
        bindung und das Engagement im Unternehmen misst. Für Entwicklerteams liefert es zwar Stimmungs­trends,
        doch es fehlen auf Entwickler zugeschnittene Kontextbasierte Daten.

  \item[TeamMood] verschickt täglich einen kurzen Stimmungs-Prompt per E-Mail, Slack oder Teams und visualisiert
        die Antworten als Verlaufsdiagramm. Die Lösung ist niedrigschwellig, erfasst jedoch keine Code- oder
        Prozessmetriken und gibt auch keine automatisierten Empfehlungen, sodass Entwickler selbst interpretieren
        müssen, was aus den Daten folgt.

  \item[Happimeter] stammt aus einem Forschungsprojekt der TU Wien und erzeugt einen „Happiness-Score“ auf Basis
        von Wearable-Sensoren wie Herzfrequenz und Aktivität. Es fehlt jedoch der Kontext zur eigentlichen
        Entwicklugsarbeit.

  \item[Code Climate Velocity] analysiert Pull-Request-Dauer, Cycle-Time und Review-Aktivität, um Teamleistung zu
        beurteilen. Die Plattform konzentriert sich vollständig auf Prozess- und Code-Metriken. Subjektive
        Zufriedenheitsdaten bleiben aussen vor, sodass der Einfluss der Stimmung auf die gemessene Performance
        unsichtbar bleibt.

  \item[GitHub Insights] bietet Dashboards zu Commits, Pull-Requests und Release-Takt. Das Tool liefert wertvolle
        Aktivitäts­statistiken, berücksichtigt jedoch keine emotionalen Faktoren.

  \item[Microsoft Viva Insights] wertet Kalendereinträge, Meetings und Kommunikationsmuster in Microsoft 365 aus,
        um Arbeitsgewohnheiten zu optimieren. Zwar erhält man Hinweise zu Meeting-Überlastung oder Fokuszeiten,
        doch wird das Wohlbefinden ausschliesslich indirekt aus Kommunikationsdaten abgeleitet, ohne jede Kopplung
        an tatsächliche Entwickler­tätigkeiten oder subjektive Gefühlslagen.
\end{description}

Engagement-Tools erfassen Stimmung, Dev-Analytics Leistung, Gesundheitsdaten, doch keine Lösung integriert alle drei
Dimensionen und leitet automatisch konkrete Verbesserungen ab. Yappi positioniert sich daher als Brückentechnologie, die

\begin{itemize}
  \item das Erfassen von Zufriedenheitsdaten in den Arbeitsprozess integriert,
  \item Daten zu Commits und Kalendererereignissen automatisch als Kontext mit einbezieht,
  \item zusätzlich Gesundheitsdaten sammelt,
  \item per KI Coach handlungsorientierte Empfehlungen liefert.
\end{itemize}

\chapter{Methoden}
\todo{Kanban erwähen}
\section{Projektmethodik}
\section{Prototypen}
\section{Proof of Concepts}

\todo{arc42 erwähen}

\chapter{Konzeptentwurf}
\section{Zugriffskontrolle über API Keys}

\todo{Kapitel evt umbenennen; Aussage soll sein die das Yappi zu Plattform wird...}

\section{Companion Apps}
\subsection{Integration in die Entwicklungsumgebung}
\subsection{Integration von Kalenderdaten}
\xeno{}
\subsection{Integration von Gesundheitsdaten}

Die folgenden Gesundheitsmetriken sind für unsere Arbeit relevant, da sie einen interessanten Einfluss
auf die Zufriedenheit von Softwareentwicklern haben.

\todo{Nochmals anschauen}
\begin{enumerate}
  \item \textbf{Schlafdauer}\\
        Untersuchungen zeigen, dass längere Schlafdauer mit höherer Arbeitszufriedenheit einhergeht:
        Männer mit mehr Schlaf berichten signifikant grössere Zufriedenheit am Arbeitsplatz als
        solche mit verkürztem Schlafpensum.

  \item \textbf{Ruheherzfrequenz}\\
        Eine erhöhte Ruheherzfrequenz spiegelt häufig chronisch erhöhte Stresslevel wider. In einer
        Querschnittsstudie war hoher Job-Strain mit erhöhter RHR assoziiert, und gleichzeitig berichten
        Beschäftigte in stark belastenden Jobs über signifikant geringere Zufriedenheit.

  \item \textbf{Stress (HRV)}\\
        Die Herzratenvariabilität (HRV) ist ein objektiver Marker für die autonome Balance: Niedrige
        HRV-Werte korrelieren konsistent mit höheren Stresslevels am Arbeitsplatz. Da hohe Stresslevels
        nachweislich die Zufriedenheit verringern, eignet sich HRV-Monitoring als indirektes Mass für
        potenzielle Unzufriedenheit.

  \item \textbf{Aktivitätsminuten und Schritte}\\
        Regelmässige moderate Bewegung, erfasst über Schritte und aktive Minuten, steht in direktem
        Zusammenhang mit höherer Jobzufriedenheit. Eine aktuelle Studie belegt, dass wöchentliche
        Freizeitaktivität signifikant positive Effekte auf die Zufriedenheit am Arbeitsplatz hat.
\end{enumerate}

\todo{zu viel?}
Diese Kennzahlen erlauben es, Zusammenhänge zwischen erholungsbezogenen Faktoren und der subjektiven
Arbeitszufriedenheit zu erkennen und so gezielte Massnahmen zur Förderung des Wohlbefindens und der
Leistungsfähigkeit abzuleiten.

\section{Yappi Coach}
\section{Konzeptevaluation}
\xeno{Fragebogen}

\chapter{Implementierung}
\section{Zugriffskontrolle über API Keys}

\todo{Quelle für spring Securtiy Architektur}
https://docs.spring.io/spring-security/reference/servlet/architecture.html

\section{Companion Apps}
\subsection{IntelliJ IDEA Companion}
\subsection{Calendar Companion}
\xeno{}
\subsection{Health Companion}
\section{Deployment}
\xeno{}
\todo{nicht sicher ob es auch ein entsprechendes Kapitel im Konzeptdesign benötigt}
\todo{UML Deployment Diagramm}

\chapter{Evaluation}
\section{Beantwortung der Fragestellung}
\xeno{}

\chapter{Diskussion}

\printbibliography

\end{document}
