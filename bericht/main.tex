\documentclass[12pt,a4paper]{report}

\usepackage[backend=biber, style=ieee]{biblatex}
\addbibresource{literature.bib}

\usepackage[utf8]{inputenc}
\usepackage[T1]{fontenc}
\usepackage[ngerman]{babel}
\usepackage{lmodern}
\usepackage{enumitem}

\usepackage{xcolor}
\newcommand{\todo}[1]{\colorbox{red}{\textbf{TODO: #1}}\\}
\newcommand{\question}[1]{\colorbox{yellow}{\textbf{QUESTION: #1}}\\}
\newcommand{\xeno}[1]{\colorbox{pink}{\textbf{TODO XENO: #1}}\\}
\newcommand{\gideon}[1]{\colorbox{green}{\textbf{TODO GIDEON: #1}}\\}

\begin{document}

\begin{titlepage}
  \centering
  {\huge \textbf{Yappi - Developer Happiness} \par}
  {\large IP5 Project \par}
  \vspace{0.5cm}
  {Windisch, August 2025 \par}
  \vspace{0.5cm}

  \begin{tabular}{@{}ll@{}}
    \textbf{Studenten:}    & Xeno Isenegger, Gideon Monterosa \\
    \textbf{Fachbetreuer:} & Norbert Seyff, Nitish Patkar
  \end{tabular}

  \vfill
  {Fachhochschule Nordwestschweiz, Hochschule für Informatik \par}
\end{titlepage}

\chapter*{Abstract}
\newpage

\tableofcontents
\newpage

\listoffigures
\newpage

\listoftables
\newpage

\chapter{Einleitung}

\section{Motivation}
\todo{aus Project Agreement: zu überarbeiten}

Die Zufriedenheit von Entwicklerinnen und Entwicklern wird, wenn überhaupt, meist nur anhand der
Menge ihrer geleisteten Arbeit gemessen. Dabei entstehen zwangsläufig Defizite, und ein halbjährli-
ches Mitarbeitergespräch erweist sich oft als wenig wirksame Massnahme zur Problemlösung.

Dieses Projekt baut auf einer bestehenden Arbeit auf, in der eine Plattform zur Erfassung der Ent-
wicklerzufriedenheit entwickelt wurde. Die Webapplikation Yappi ermöglicht es Entwicklerinnen und
Entwicklern, ihre Zufriedenheit mit ihrer Arbeit und ihrer aktuellen Situation fortlaufend zu bewerten.
Yappi erfasst emotionale Faktoren wie Happiness sowie weitere Zufriedenheitsindikatoren. Zusätz-
lich können spezifische Aufgaben und Arbeitstypen individuell bewertet werden. Die erhobenen Da-
ten werden anonym auf Teamebene analysiert, um ein fundiertes Verständnis für die Stimmung in-
nerhalb der Teams zu gewinnen.

Entwicklerinnen und Entwickler haben die Möglichkeit, ihre Zufriedenheit für verschiedene Teams zu
erfassen, wodurch gezielte Analysen ermöglicht werden. Unternehmen erhalten dadurch wertvolle
Einblicke, um das Arbeitsumfeld gezielt zu verbessern.

Unser Projekt baut auf Yappi auf und zielt darauf ab, die Erfassung der Zufriedenheit weiter zu opti-
mieren. Es wird untersucht, wie die Daten noch präziser erfasst und ausgewertet werden können,
um langfristige Verbesserungen zu unterstützen. Diese Arbeit dient als Grundlage für ein weiterfüh-
rendes Forschungsprojekt, das sich vertieft mit der Entwicklerzufriedenheit auseinandersetzt und
zusätzliche Erkenntnisse gewinnen soll.

\section{Ziele und Vision}

\todo{Text aus dem Project Agreement noch zu überarbeiten}

Yappi wird zu einer umfassenden Plattform weiterentwickelt, die nicht nur die Zufriedenheit misst,
sondern sich nahtlos in den Arbeitsalltag integriert und wertvolle Handlungsempfehlungen liefert.
Dazu werden folgende Kernaspekte umgesetzt:

\subsubsection{Produktivitätsfaktoren identifizieren}

Durch eine tiefere Analyse von Zufriedenheitsindikatoren sollen zentrale Faktoren ermittelt werden,
die sich positiv oder negativ auf die Produktivität und das Wohlbefinden von Entwicklerinnen und
Entwicklern auswirken. Diese Erkenntnisse werden genutzt, um Vorschläge zu Verbesserungsmass-
nahmen abzuleiten.

\subsubsection{Integration in den Arbeitsprozess}

Yappi soll sich direkt in bestehende Arbeitsabläufe einfügen, um die Erfassung der Zufriedenheit
möglichst intuitiv und effizient zu gestalten. Dies kann durch verschiedene Schnittstellen und Erwei-
terungen erfolgen, die eine nahtlose Interaktion ermöglichen.

\subsubsection{Erweiterung um kontextbezogene Daten}

Um ein umfassenderes Bild der Arbeitszufriedenheit zu erhalten, können weitere Einflussfaktoren
berücksichtigt werden. Dazu gehören beispielsweise arbeitsbezogene Rahmenbedingungen oder
individuelle Gesundheits- und Belastungsindikatoren. Diese Daten sollen helfen, ein besseres Ver-
ständnis für langfristige Trends und Zusammenhänge zu entwickeln.

\subsubsection{Intelligente Analyse und Handlungsempfehlungen}

Durch die Integration von AI schnittstellen können gezielte Analysen erstellt und individualisierte
Empfehlungen abgeleitet werden. Dies kann sowohl auf individueller als auch auf Teamebene erfol-
gen, um nachhaltige Verbesserungen im Arbeitsumfeld zu fördern.

\subsubsection{Fazit}

Mit diesen Erweiterungen wird Yappi zu einem essenziellen Bestandteil des Entwickleralltags. Es
bietet nicht nur eine präzisere Erfassung der Zufriedenheit, sondern liefert auch wertvolle Einblicke
und Handlungsempfehlungen, um die Arbeitsbedingungen nachhaltig zu verbessern. Unternehmen
erhalten fundierte Analysen und können gezielt Massnahmen ergreifen, um eine motivierte und pro-
duktive Entwicklergemeinschaft zu fördern.

\section{Fragestellungen}

\todo{Text aus dem Project Agreement noch zu überarbeiten}

\begin{enumerate}[label=\Alph*.]
  \item Durch welche Technologien und Schnittstellen kann Yappi erweitert werden,
        um ein reibungsloses und einfaches Erfassen von Zufriedenheitsdaten zu ermöglichen?
        \begin{enumerate}[label=\alph*.]
          \item Entwicklung von Entwickler-Tool-Plugins, die nahtlos in bestehende Arbeitsumgebungen
                integriert werden können, um die Nutzung von Yappi angenehmer und effizienter zu gestalten.
                Diese Plugins sollen Entwicklern ermöglichen, direkt in ihrer bevorzugten Umgebung Feedback
                zu erfassen, ohne den Arbeitsfluss zu unterbrechen. Integration von Yappi in verschiedene
                Plattformen und Tools wie Webbrowser, IntelliJ, Microsoft Teams und Outlook.
        \end{enumerate}
  \item Wie können Gesundheitsdaten in die Auswertung der Entwicklerzufriedenheit einfliessen?
        \begin{enumerate}[label=\alph*.]
          \item Direkte Anbindung der Gesundheitsdaten-API, um relevante Gesundheitsmetriken wie
                Herzfrequenz, Schlafqualität oder Stresslevel automatisch in die Analyse der
                Entwicklerzufriedenheit zu integrieren. Dies ermöglicht eine genauere Einschätzung des
                Wohlbefindens und potenzieller Belastungsfaktoren.
        \end{enumerate}
  \item Wie kann Yappi Teams und Entwickler dabei unterstützen, aus den erfassten
        Zufriedenheitsdaten Handlungsempfehlungen abzuleiten, um die Zufriedenheit und Produktivität
        von Entwicklern zu erhöhen?
        \begin{enumerate}[label=\alph*.]
          \item Entwicklung eines Yappi Coach, der anhand einer detaillierten Analyse der erfassten
                Daten gezielte Tipps zur Verbesserung der Arbeitsweise gibt. Beispielsweise könnte der Coach
                darauf hinweisen, dass Meetings nicht länger als 1,5 Stunden dauern sollten, da längere
                Sitzungen die Zufriedenheit und Konzentration der Entwickler negativ beeinflussen können.
          \item Integration von KI-gestützten Diensten, die auf Basis der gesammelten Gesundheitsdaten
                sowie Zufriedenheits- und Produktivitätsmetriken individuelle Massnahmen vorschlagen. Diese
                KI-gestützten Empfehlungen können Teams dabei helfen, gezielt Optimierungen vorzunehmen, um
                die Arbeitsbedingungen und die Effizienz der Entwickler nachhaltig zu verbessern.
        \end{enumerate}
\end{enumerate}

\chapter{Hintergrund}
\section{Ausgangslage}
\section{Aktueller technischer Stand}
\section{Parallele Entwicklung auf gemeinsamer Codebasis}
\section{Stakeholder}
\todo{unterkapitel für den Stand von Yappi vor dem Projekt}

\chapter{Methoden}
\todo{Kanban erwähen}
\section{Projektmethodik}
\section{Prototypen}
\section{Proof of Concepts}
\todo{arc42 erwähen}

\chapter{State of the Art}
\section{Definition von Entwiklerzufriedenheit}

Entwicklerzufriedenheit wird in der Literatur als Balance zwischen positiven und negativen Erlebnissen bei der
Arbeit definiert. Darunter versteht man eine Sequenz von Erfahrungen, bei der häufige positive Emotionen ein
hohes Glücksgefühl erzeugen und häufige negative Erfahrungen das Gegenteil bewirken \cite{sadowski_happiness_2019}.
Auch Industriequellen fassen Entwicklerzufriedenheit als subjektives Wohlbefinden in Bezug auf Arbeitsinhalte
und -umfeld auf, d.h. als Mass für Zufriedenheit, Freude oder innere Zufriedenheit bei der Arbeit \cite{zenhub_2022_nodate}.
Zufriedene Entwickler empfinden demnach mehr Arbeitsfreude und Inhaltlichkeit in ihrer Rolle, was eng mit
der Arbeitsmotivation und dem Engagement bei der Arbeit verknüpft ist \cite{franca_motivation_2020}.

\todo{evt braucht es hier eine citation oder fussnote für den originalen Begriff von flow}
\todo{Ich bin mir unsicher ob die citation in diesem Absatz sauber genug ist}
Eng verwendt mit der Zufriedenheit ist der Begriff \textbf{Flow}. In Anlehnung an Csikszentmihalyis Konzept beschreibt
Flow einen Zustand von völliger Vertiefung und hohen Fokus beim Programmieren. Flow tritt dann auf, wenn die Anforderungen
einer Aufgabe im Gleichgewicht mit dem Fähigkeiten des Entwicklers stehen, wodurch man in einen Zustand von intensiver 
Konzentration gelangt. Zufriedene Entwickler gelangen einfacher in einen anhaltenden Flow-Zustand. Unzufriedenheit
hingegen unterbricht diesen Flow, was zu Frustration führt und Schwierigkeiten führt, nach Unterbrechungen wieder in
eine Aufgabe zurückzufinden. Teilnehmer einer Untersuchung berichten, negative Erlebnisse reissen einen aus dem Flow
Zustand und machen es schwer, die Arbeit wieder aufzunehmen \cite{sadowski_happiness_2019}.

\todo{maybe absatz über burnout}

Motivation und Zufriedenheit hängen eng zusammen, sind aber konzeptionell unterscheidbar. Motivierte Entwickler sind
zeigen hohes Engagement und Fokus auf ihre Aufgaben, während Zufriedenheiteher durch allgemeines Wohlbefinden und gute
Laune charakerisiert ist. Faktoren wie Autonomie, Kompetenzerleben und Zugehörigkeitsgefühl steigern die intrinsische 
Motivation von Entwicklern, was sich positiv auf ihre Zufriedenheit auswirkt. Zufriedenheit ist zugleich das Ergebnis und
die Voraussetzung von Motivation, zufriedenere Entwickler weisen in der Regel eine höhere Antriebskraft auf, was wiederum ihre
Arbeitszufriedenheit weiter stärkt \cite{franca_motivation_2020}.

Schliesslich spielt auch das Team- und Organisationsklima eine fundamentale Rolle. Eine offene, unterstützende Kultur steigert
nachweislich die zufriedenheit von Entwicklern. Der DORA Report misst die Leistungsfähigkeit von Softwareentwicklungsteams anhand
von vier Schlüsselkennzahlen: Deployment Frequency, Lead Time for Changes, Change Failure Rate und Time to Resore Service. Der DORA
Report von Google basiert auf umfangreichen Wissenschaftlichen Studien und gilt als Branchenstandart. Der report von 2024 betont,
dass Teams mit stabilem, ermutigendem Umfeld bessere Ergebnisse erzielen \cite{google_dora_2024}. Positive Emotionen und ein 
Zugehörigkeitsgefühl im Team fördern den Gruppenzusammenhalt, was wiederum die Teamleistung von Teammitgliedern verbessert. 
Umgekehrt können Umgekehrt können toxische Kulturen oder ständig wechselnde Prioritäten die Zufriedenheit und Motivation 
untergraben, was sich negativ auf die Leistung auswirkt \cite{sadowski_happiness_2019}.

Somit unterstreichen sowohl akademische als auch industrielle Befunde: Entwicklerzufriedenheit entsteht in einem komplexen
Zusammenspiel aus individuellen Faktoren (Flow, Motivation, ...) und Umfeldfaktoren (Team- und Organisationsklima, Arbeitskultur, 
...).

\section{Stand der Forschung und verwandte Arbeiten}

\question{Definitionen für Erfolgsparameter einfügen?}
In den letzen Jahren haben zahlreiche Studien den Zusammenhang zwischen der Entwicklerzufriedenheit und Erfogsparametern wie 
Produktivität, Codequalität und Mitarbeiterbindung untersucht. Bei einer grossangelegten Studie wurden 317 Softwareentwickler
befragt und dabei 42 Konsequenten von Unzufriedenheit sowie 32 Konsequenzen von Zufriedenheit beim Programmieren identifiziert.
Die Ergebnisse Zeigen, dass Entwicklerzufriedenheit messbare Auswirkungen auf den Enwicklungsprozess, die erzeugten 
Software-Artefakte und das Wohlbefinden der Person hat. So führt Unzufriedenheit zu einer Reihe negativer Effekte: verzögerte
Prozessabläufe, nachlässige Arbeitsweise und häufige unterbrechungen des Flows wurden als typische Folgen von negativer Stimmung
genannt. Unzufriedene Entwickler berichten von langen langen Verzögerungen oder Qualitätsproblemen, weil Frustration sie aus dem
Konzept brachte. Zufriedenheit hingegen wirkt sich positiv aus: Zufriedene Entwickler zeigen bessere Problemlösungsfähigkeiten,
höhere Konzentration, berichten von einem anhaltenden Flow Zustand und lernen schneller. Die Ergebnisse legen nahe, dass 
Zufriedenheit die Codequalität begünstigt. Zufriedene Entwickler treffen sorgfältigere langfristige Entscheidungen. Ein Teilnehmer
beschrieb, er dokumentiert seinen Code gründlicher und achten stärker auf Wartbarkeit, wenn er Zufrieden ist 
\cite{graziotin_what_2018}.

Neben akademischen Studien liefern auch Industireumfragen und Community-Studien wertvolle Einblicke. Der jährliche Stack Overflow
Developer Survey etwa spiegelt wieder, dass weiterhin Verbesserungsbedarf besteht. laut der Umfrage 2024 bezeichen sich nur rund 20\%
der Entwickler als wirklich zufrieden in ihrem Job, während etwa 80\% unzufrieden oder "gelassen" (complacent) sind. Diese Zahl
unterstreicht, dass ein grosser Teil der Entwicklergemeinschaft nicht glücklich im Arbeitsumfeld ist. Als Hauptgründe werden oft
Faktoren wie schlechte Work-Life-Balance, zu viele Meetings oder fehlende Anerkennung genannt \cite{stackoverflow_survey_2025}. Eine
Untersuchung von Zenhub kam zu ähnlichen Ergebnissen: Zwar gaben die meisten befragten Entwickler an, überwiegend zufrieden zu sein,
doch lediglich 31\% fühlten sich äusserst zufrieden in ihrer aktuellen Arbeitssituation \cite{zenhub_2022_nodate}. 

Darüber hinaus tragen Anerkennung und Sinnhaftigkeit der Arbeit entscheidend zur Bindung bei. Wenn Entwicklerinnen und Entwickler
stolz auf die Qualität ihrer Projekte sind und regelmässig Wertschätzung für ihre Arbeit erhalten, steigt sowohl ihre Zufriedenheit
als auch ihre Loyalität gegenüber dem Unternehmen \cite{sadowski_happiness_2019,graziotin_what_2018}. Die empirischen Befunde deuten
somit klar darauf hin, dass Entwicklerzufriedenheit kein rein „weiches“ Thema ist, sondern messbare Auswirkungen auf Produktivität,
Codequalität und Personalbindung hat. Zufriedene Entwickler arbeiten effizienter, treffen sorgfältigere Entscheidungen und bleiben
ihrem Team länger erhalten, während Unzufriedenheit mit erhöhten Kosten für Projekte und Rekrutierung verbunden ist.

\todo{evt. Methoden und Instrumente zur Messung der Entwicklerzufriedenheit}
\todo{evt. Technologische und methodische Ansätze in verwandten Arbeiten}
\todo{Forschungslücken}

\section{Bestehende Lösungen und Wettbewerbsanalyse}

Der Markt bietet bereits verschiedene Tools und Plattformen, die Teilaspekte der Entwicklerzufriedenheit adressieren. Im Folgenden
werden einige repräsentative bestehende Lösungen vorgestellt und hinsichtlich ihres Fokus und ihrer Lücken bewertet:

\begin{description}
  \item[Officevibe:] Ein Software-as-a-Service-Tool, das wöchentliche Pulse-Umfragen an Mitarbeitende verschickt. Ziel ist es,
    Engagement und Stimmung im Unternehmen kontinuierlich zu messen. Officevibe bietet anonyme wöchentliche Kurzbefragungen per
    E-Mail oder Chat an, deren Ergebnisse in übersichtlichen Dashboards für Teamleiter aufbereitet werden. Entwicklerteams erhalten
    dadurch Stimmungs-Trendkurven und allgemeines Mitarbeiterfeedback. Allerdings ist Officevibe eher generisch auf
    Mitarbeiterengagement ausgerichtet und liefert keine speziell auf Entwickler zugeschnittenen Kontextdaten, z.B. werden keine
    technischen Metriken aus der Softwareentwicklung einbezogen \cite{courier_officevibe_2025}.

  \item[TeamMood:] Ein schlankes Stimmungsbarometer für Teams, das täglich eine einfache Mood-Abfrage durchführt. TeamMood sendet
    jedem Teammitglied jeden Tag einen kurzen Prompt (per E-Mail, Slack, microsoft teams etc.), in dem die Person mit einem Klick
    ihre aktuelle Stimmung angibt. Die Antworten werden als anonymer Team-Stimmungsverlauf visualisiert, was es erlaubt, Trends über
    die Zeit zu erkennen. Die Hürde zur Teilnahme ist sehr niedrig (niedrigschwelliges Feedback). Jedoch erfasst TeamMood keine
    technischen Prozessmetriken (wie Code-Commits, Buildzeiten o.ä.) und bietet keine automatisierten Empfehlungen. Das bedeutet,
    dass Entwickler und Manager die Stimmungsverläufe selbst interpretieren und Massnahmen ableiten müssen, ohne direkte
    Handlungsempfehlungen durch das Tool \cite{revelo_teammood_2025}.

  \item[Happimeter:] Hervorgegangen aus einem Forschungsprojekt (u.a. TU Wien und MIT) setzt Happimeter auf Wearable Sensoren, um
    einen persönlichen Happiness-Score zu bestimmen. Entwickler tragen z.B. eine Smartwatch mit der Happimeter-App, die
    kontinuierlich physiologische Daten wie Herzfrequenz, Bewegung oder Schlaf erfasst. Ein Machine-Learning-Modell sagt daraus
    die aktuelle Stimmung bzw. den Stresslevel der Person voraus. Auf diese Weise sollen objektive Gesundheitsmetriken mit dem
    subjektiven Wohlbefinden verknüpft werden. Der Ansatz liefert interessante biometrische Einblicke (z.B. Stressspitzen während
    der Arbeit), jedoch fehlt der direkte Bezug zur eigentlichen Entwicklungsarbeit. Happimeter weiss nichts über Tasks, Code oder
    Arbeitskontext, sodass die sensorbasierten Glücks-Werte ohne diesen Kontext schwer zu interpretieren sind
    \cite{budner_making_2017}.

  \item[Code Climate Velocity:] Code Climate Velocity: Eine Datenplattform, die Entwicklungsmetriken aus Git-Repositories analysiert,
    um die Team-Performance zu bewerten. Velocity konzentriert sich voll auf quantitative Software-Engineering-Kennzahlen. Es misst
    etwa die Durchlaufzeit von Pull Requests, die Commit-Frequenz, die Review-Geschwindigkeit und diverse weitere Metriken der
    Entwicklungspipeline. Auch Industrie-Standards wie die vier DORA-Metriken (Deployment Frequency, Lead Time for Changes, Change
    Failure Rate, Mean Time to Restore Service) sind integriert. Dadurch erhalten Führungskräfte einen detaillierten Blick auf
    Code-Qualität, Liefergeschwindigkeit und Prozess-Effizienz. Allerdings fehlen subjektive Zufriedenheitsdaten vollständig, die
    menschliche Stimmungslage der Entwickler wird nicht erfasst. Etwaige Einflüsse von Motivation oder Frustration auf die
    gemessenen Leistungsindikatoren bleiben somit unsichtbar \cite{infoworld_codeclimate_2023}.

  \item[GitHub Insights:] Als integrierter Teil von GitHub bietet Insights grundlegende Analysen zur Repository-Aktivität. In jedem
    GitHub-Repository steht ein Insights-Dashboard zur Verfügung, das Statistiken zu Commits, Pull-Request-Aktivitäten,
    Issue-Verläufen und Release-Frequenzen visualisiert. Teams können so ihre Entwicklungsaktivität und Geschwindigkeit verfolgen
    (“Wie viele PRs werden pro Woche gemerged? Wie oft wird deployed?” etc.). Diese Metriken sind wertvolle Aktivitätsstatistiken,
    berücksichtigen jedoch keine emotionalen Faktoren. GitHub Insights liefert also Kennzahlen zur Produktivität, blendet aber das
    Stimmungsbild der Entwickler aus. Etwa ob eine Phase hoher Commit-Rate auf Überstunden und Stress zurückzuführen ist, bleibt
    unklar \cite{axify_git_2025}.

  \item[Microsoft Viva Insights:] Viva Insights ist Teil von Microsoft 365 und zielt darauf ab, durch Analyse von Arbeitsmustern die
    Produktivität und das Wohlbefinden von Mitarbeitenden zu verbessern. Die Plattform wertet vor allem Kalender- und
    Kommunikationsdaten aus (z.B. E-Mail- und Teams-Nutzung, Meeting-Häufigkeit). Entwickler erhalten z.B. Hinweise, wenn
    Meeting-Überlast droht, oder Vorschläge, regelmässige Fokuszeiten für ungestörtes Arbeiten einzuplanen. Führungskräfte sehen
    aggregierte Team-Insights, etwa ob viele Überstunden anfallen oder wenig Konzentrationsphasen vorhanden sind. Zwar gibt Viva
    nützliche Empfehlungen (z.B. “Schützen Sie wöchentlich 4 Stunden Fokuszeit” oder “Vermeiden Sie Meetings über 1 Stunde”).
    Allerdings werden Wohlbefinden und Zufriedenheit nur indirekt aus den Verhaltensdaten abgeleitet, eine direkte Erfassung der
    Gefühlslage oder ein Bezug zu konkreten Entwickler-Tätigkeiten (Code, Tickets etc.) fehlt vollständig. Somit bleibt die
    emotionale Dimension in Viva Insights eher implizit und generalisiert \cite{zachminers_introduction_nodate}.
\end{description}

Bestehende Lösungen decken jeweils nur einzelne Aspekte der Entwicklerzufriedenheit ab. Engagement-Tools erfassen Stimmungsdaten,
stellen jedoch keinen Bezug zu technischen oder gesundheitlichen Kontextinformationen her. Engineering-Analytics-Tools analysieren
Leistungskennzahlen, berücksichtigen jedoch die emotionale Dimension nicht. Gesundheits-Tracker wiederum messen physiologische Daten,
ohne diese mit der konkreten Entwicklungsarbeit zu verknüpfen. Eine Plattform, die emotionale Faktoren, technischen Kontext und
physisches Wohlbefinden gemeinsam erfasst und daraus konkrete Handlungsempfehlungen ableitet, ist derzeit nicht verfügbar.

\chapter{Konzeptentwurf}

Die im vorhergehenden Kapitel dargestellte Analyse bestehender Lösungen verdeutlicht, dass es derzeit keine Plattform gibt, welche
emotionale Faktoren, technischen Kontext und physisches Wohlbefinden integriert erfasst und daraus konkrete Handlungsempfehlungen
ableitet. Yappi soll genau diese Lücke schliessen und eine einheitliche, praxisorientierte Lösung bieten. Ziel ist es, die
Entwicklerzufriedenheit umfassend zu erfassen und die gewonnenen Erkenntnisse in konkrete Massnahmen zu überführen. Die geplanten
Kernfunktionen umfassen:

\begin{description}
  \item[Integriertes Erfassen im Arbeitsfluss:] Yappi lässt sich nahtlos in den täglichen Entwicklungsprozess einbetten. Durch 
    Plugins für Entwicklungsumgebungen, Browser-Erweiterungen oder Integrationen in Kollaborationstools können Entwickler ihre
    Zufriedenheit direkt in ihrer Arbeitsumgebung erfassen, ohne den Kontext zu wechseln. Dies erhöht die Teilnahmebereitschaft
    und stellt sicher, dass Feedback leicht und unmittelbar gegeben werden kann ohne den Arbeitsfluss zu unterbrechen.
  \item[Automatischer Kontext durch Prozessdaten:] Neben manuellen Stimmungsangaben bezieht Yappi automatisch Kontextinformationen
    aus dem Arbeitsprozess ein. Beispielsweise können Commit-Daten aus Git und Kalendereinträge herangezogen werden, um die Stimmung
    in Bezug zu objektiven Ereignissen zu setzen. So liesse sich erkennen, ob z.B. eine Häufung negativer Stimmungswerte mit vielen
    Meetings korreliert.
  \item[Erweiterung um Gesundheitsmetriken:] Um ein umfassenderes Bild des Wohlbefindens zu erhalten, soll Yappi optional auch
    Gesundheitsdaten einbeziehen. Dies kann durch die Anbindung an gängige Gesundheits-APIs oder Wearables (ähnlich Happimeter)
    erfolgen. Die so gewonnenen zusätzlichen Datenpunkte ermöglichen es, langfristige Entwicklungen zu erkennen und diese im
    Zusammenhang mit Prozess- sowie Zufriedenheitsdaten zu analysieren.
  \item[KI-gestützter Coach mit Handlungsempfehlungen:] Über die reine Datenerfassung hinaus soll Yappi einen intelligenten
    Empfehlungsdienst bereitstellen. Dieser Yappi-Coach analysiert die erfassten Zufriedenheitsdaten, Produktivitätskennzahlen und
    gegebenenfalls Gesundheitsdaten mittels KI-gestützter Verfahren. Auf dieser Grundlage werden gezielte, evidenzbasierte
    Empfehlungen für einzelne Entwicklerinnen und Entwickler sowie für Teams generiert. Beispielsweise könnte der Coach vorschlagen,
    die Dauer von Team-Meetings auf höchstens 1,5 Stunden zu begrenzen, wenn längere Sitzungen mit sinkender Zufriedenheit
    einhergehen. Ebenso kann er empfehlen, nach vier Stunden konzentrierter Entwicklungsarbeit eine Pause einzulegen, sofern die
    Analyse zeigt, dass dies die Zufriedenheit steigert. Ziel ist es, konkrete Handlungsimpulse zu geben, die sowohl das
    individuelle Wohlbefinden als auch die Teamproduktivität verbessern.
\end{description}

Durch die Kombination dieser Aspekte entsteht mit Yappi eine zentrale Plattform, die kontinuierliches Stimmungs-Tracking,
automatisierte Kontextdaten aus dem Arbeitsprozess, Gesundheitsmetriken und KI-gestützte Handlungsempfehlungen integriert. Ziel ist
es, die Entwicklerzufriedenheit nicht nur präzise zu erfassen, sondern die gewonnenen Erkenntnisse unmittelbar in nachhaltige
Verbesserungen des Arbeitsalltags zu überführen. Unternehmen erhalten damit eine fundierte Entscheidungsgrundlage, um gezielt
Massnahmen zur Förderung einer motivierten, gesunden und leistungsfähigen Entwicklergemeinschaft umzusetzen. Die nachfolgenden
Abschnitte greifen diese Kernaspekte auf und verknüpfen sie mit weiteren Konzeptbestandteilen, um ein ganzheitliches Lösungsdesign
zu entwickeln.

\section{Zugriffskontrolle über API Keys}

\todo{Kapitel evt umbenennen; Aussage soll sein die das Yappi zu Plattform wird...}

\section{Companion Apps}
\subsection{Integration in die Entwicklungsumgebung}
\subsection{Integration von Kalenderdaten}
\xeno{}

Ein vollständiges erfassen der Entwicklerzufriedenheit erforder die Berücksichtigung aller Arbeitsaspekte,
insbesondere auch Besprechungen (Meetings). Empirische untersuchungen zeigen, dass Entwickler durchschnittlich zwischen
10,9 Stunden und 16,5 Stunden pro Woche in Besprechungen verbringen. Ein erheblicher Teil dieser Zeit wird insbesondere
bei grossen oder ungünstig terminierten Besprechungen als wenig produktiv bewertet.
Eine hohe Meetingfrequenz kann den Arbeitstag fragmentieren und den Flow unterbrechen \cite{stray_understanding_2020, meyer_today_2021}.

Ziel dieses Konzeptentwurf ist die Entwicklung einer Kalendererweiterung für Yappi, welche durch das Integrieren des Persönlichen Kalender
in die bestehenden Webanwendung Yappi und einer Companion-App welche durch eine Browser-Erweiterung einsicht in die Meetings der Entwickler ermöglicht.
Die Erweiterung soll nach Ende eines Meetings automatisch, den Entwickler nach Feedback abfragen und die Ergebnisse davon in Yappi für weitere Auswertungen bereitstellen.

\subsubsection{Messung der Meetingqualität}
\xeno{rework mit quelle}
Zur Bewertung werden sieben Kriterien definiert. Diese sind in Anlehnung an Best Practices und gängige Meeting-Umfrageinstrumente ausgewählt:

\begin{enumerate}
    \item \textbf{Zielklarheit und Relevanz} \\
    Prüft, ob die Meetingziele klar formuliert und relevant waren. Eine präzise Agenda und eindeutige Zieldefinition steigern die Produktivität.

    \item \textbf{Inhaltliche Tiefe und Verständlichkeit} \\
    Bewertet, ob Themen angemessen tief und zugleich verständlich behandelt wurden. Unnötige Detailfülle ist zu vermeiden.

    \item \textbf{Zeitmanagement} \\
    Umfasst Pünktlichkeit beim Start und Einhaltung der geplanten Dauer. Studien empfehlen, Meetingzeiten bewusst zu verkürzen (z. B. 25 statt 30 Minuten), um Effizienz zu erhöhen.

    \item \textbf{Moderation und Beteiligung} \\
    Erfasst die Qualität der Moderation und die aktive Einbindung der Teilnehmenden. Eine hohe Beteiligungsquote gilt als Indikator für Interaktivität.

    \item \textbf{Ergebnisorientierung} \\
    Misst, ob konkrete Entscheidungen oder Aufgaben resultierten. Die Anzahl abgeschlossener Aktionspunkte kann als Kennzahl dienen.

    \item \textbf{Allgemeine Zufriedenheit} \\
    Ermittelt das subjektive Gesamturteil, beispielsweise auf einer Skala von 1 bis 10.

    \item \textbf{Meetingdauer} \\
    Vergleicht geplante mit tatsächlicher Dauer und bewertet die Angemessenheit.
\end{enumerate}

Diese Kriterien ermöglichen ein umfassendes Bild der Meetingqualität und liefern Ansatzpunkte für Verbesserungen.

\subsubsection{Kalenderintegration auf Basis offener Standards}

Zur automatischen Erkennung relevanter Meetings wird eine Anbindung an das Kalendersysteme des Entwickler vorgesehen.
Wichtig dabei ist es einen Offenen Standard zu verwenden um eine möglichst hohe Kompatibilität mit Kalendersystemen zu schaffen.
Grundlage dazu bildet das weit verbreitete iCalendar-Format (ICS), das von nahezu allen gängigen Plattformen unterstützt wird.
Das Konzept sieht vor, dass der Nutzer in Yappi auf seinem Profil den zugriff auf den Kalender hinterlegen kann.
Regelmässig werden die auf einen vom Nutzer im Kalender erstellten Kalendereinträge in die Datenbank synchronisiert.
Auf dieser Basis kann die Anwendung automatisch identifizieren, wann ein Meeting stattfindet.
Dadurch lassen sich Feedback-Anfragen unmittelbar nach dem Ende einer Besprechung auslösen, ohne dass der Nutzer manuell eingreifen muss.

Die Synchronisation des Kalender erfolgt ausschliesslich lesend, um Eingriffe in persönliche Kalender zu vermeiden.
Alle übermittelten Kalenderinformationen werden vertraulich behandelt und ausschliesslich für den definierten Zweck genutzt.
Informationen zu Kalendereinträge sind nur für den Nutzer persönlich einsehbar.

Durch diese Anbindung entfällt das manuelle Erfassung von Kalendereinträgen. Meetings werden automatisch erkannt,
Änderungen übernommen und die Nutzer benachrichtigt. So entsteht ein nahtloser integrierter feedback flow um die
Zufriedenheitsdaten von Meetings für die Entwickler zu erheben.

\subsubsection{Companion-App als Feedback-Trigger}

Die Companion-App wird als Browsererweiterung umgesetzt, um eine direkte Interaktion mit den Entwicklern zum richtigen
Zeitpunkt zu ermöglichen. Nach dem Ende eines im Kalender erfassten Meetings erhält der Nutzer eine Benachrichtigung,
die zur Abgabe einer kurzen Bewertung auffordert. Diese zeitnahe Erhebung stellt sicher, dass Eindrücke und Wahrnehmungen
noch frisch sind und die Datenqualität hoch bleibt. Die Interaktion erfolgt freiwillig, um Befragungsmüdigkeit zu vermeiden,
und ist so gestaltet, dass die Beantwortung wenige Sekunden benötigt. Durch die Integration in den Browser wird kein
zusätzlicher Systemwechsel nötig, wodurch die Hemmschwelle zur Teilnahme sinkt.

\subsubsection{Datenverarbeitung und Auswertung}

Die erhobenen Feedbackdaten werden ausschliesslich auf dem Yappi-Server verarbeitet.
Vor der Auswertung erfolgt eine Anonymisierung, um Rückschlüsse auf einzelne Personen zu verhindern.
Für jedes Meeting werden aggregierte Kennzahlen, wie Durchschnittswerte der sieben Kriterien oder Verteilungen der Bewertungen, berechnet.
Diese werden nur dann angezeigt, wenn eine vordefinierte Mindestanzahl an Rückmeldungen vorliegt.
Neben der Auswertung einzelner Meetings können über längere Zeiträume Trends analysiert und Optimierungsmassnahmen evaluiert werden.

\subsubsection{Integration in die Yappi Webanwendung}

Die bestehenden Funktionen von Yappi werden um einen Bereich zur Meetinganalyse ergänzt.
Entwickler sehen hier ihre eigenen Feedbackabgaben und offene Anfragen.
Auf der Teamebene kann auf aggregierte, anonymisierte Auswertungen zu den Meetings ihres Teams zugegriffen werden.
Die Konfiguration der Kalenderintegration und der Verbindung zur Companion-App wird im Nutzerprofil zentral verwaltet.
Dadurch bleibt die Erweiterung vollständig in die bestehende Systemarchitektur eingebettet und benötigt keine separaten Plattformen.

\subsection{Integration von Gesundheitsdaten}

Zusätzlich zur Meetinganalyse kann Yappi um die Integration von Gesundheits- und Belastungsdaten erweitert werden,
sofern der Nutzer dies explizit freigibt.
Mögliche Datenquellen sind Wearables oder manuell erfasste Angaben zu Schlaf, Bewegung oder Arbeitsbelastung.
Durch die Kombination von Meetingfeedback und Gesundheitsindikatoren lassen sich Korrelationen erkennen,
beispielsweise zwischen hoher Meetingdichte und erhöhter Ermüdung.
Diese Informationen können genutzt werden, um Arbeitsprozesse gesundheitsförderlich zu gestalten.


Die folgenden Gesundheitsmetriken sind für unsere Arbeit relevant, da sie einen interessanten Einfluss auf die Zufriedenheit von
Softwareentwicklern haben.

\todo{Nochmals anschauen}
\begin{enumerate}
  \item \textbf{Schlafdauer}\\
        Untersuchungen zeigen, dass längere Schlafdauer mit höherer Arbeitszufriedenheit einhergeht:
        Männer mit mehr Schlaf berichten signifikant grössere Zufriedenheit am Arbeitsplatz als
        solche mit verkürztem Schlafpensum.

  \item \textbf{Ruheherzfrequenz}\\
        Eine erhöhte Ruheherzfrequenz spiegelt häufig chronisch erhöhte Stresslevel wider. In einer
        Querschnittsstudie war hoher Job-Strain mit erhöhter RHR assoziiert, und gleichzeitig berichten
        Beschäftigte in stark belastenden Jobs über signifikant geringere Zufriedenheit.

  \item \textbf{Stress (HRV)}\\
        Die Herzratenvariabilität (HRV) ist ein objektiver Marker für die autonome Balance: Niedrige
        HRV-Werte korrelieren konsistent mit höheren Stresslevels am Arbeitsplatz. Da hohe Stresslevels
        nachweislich die Zufriedenheit verringern, eignet sich HRV-Monitoring als indirektes Mass für
        potenzielle Unzufriedenheit.

  \item \textbf{Aktivitätsminuten und Schritte}\\
        Regelmässige moderate Bewegung, erfasst über Schritte und aktive Minuten, steht in direktem
        Zusammenhang mit höherer Jobzufriedenheit. Eine aktuelle Studie belegt, dass wöchentliche
        Freizeitaktivität signifikant positive Effekte auf die Zufriedenheit am Arbeitsplatz hat.
\end{enumerate}

\todo{zu viel?}
Diese Kennzahlen erlauben es, Zusammenhänge zwischen erholungsbezogenen Faktoren und der subjektiven
Arbeitszufriedenheit zu erkennen und so gezielte Massnahmen zur Förderung des Wohlbefindens und der
Leistungsfähigkeit abzuleiten.

\section{Yappi Coach}
\section{Konzeptevaluation}
\xeno{Fragebogen}

\chapter{Implementierung}
\section{Zugriffskontrolle über API Keys}

\todo{Quelle für spring Securtiy Architektur}
https://docs.spring.io/spring-security/reference/servlet/architecture.html

\section{Companion Apps}
\subsection{IntelliJ IDEA Companion}
\subsection{Calendar Companion}
\xeno{}
\subsection{Health Companion}
\section{Deployment}
\xeno{}
\todo{nicht sicher ob es auch ein entsprechendes Kapitel im Konzeptdesign benötigt}
\todo{UML Deployment Diagramm}

\chapter{Evaluation}
\section{Beantwortung der Fragestellung}
\xeno{}

\chapter{Diskussion}

\printbibliography

\end{document}
